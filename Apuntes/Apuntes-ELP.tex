\documentclass[twoside]{AiTeX}



\title{Ética, Legislación y Profesión}
\author{A.L.K.}
\date{Diciembre 2021}
\begin{document}
%\datos{facultad}{universidad}{grado}{asignatura}{subtitulo}{autor}{curso}
\datos{Informática}{Universidad Complutense de Madrid}{Ingeniería informática}{Ética, Legislación y Profesión}{```Aprender''' a hacer tu trabajo de una fomra ética y legal}{Alejandro Barrachina Argudo}{2022-2023}
\portadaApuntes
\pagestyle{empty}
\tableofcontents
\pagestyle{empty}
\justify
\pagestyle{fancy}

\newpage

\section*{Control de cambios} %
\noindent\begin{tabularx}{\textwidth}{ |l|l|p{5cm}|X| }
    \hline
    \textbf{Versión} & \textbf{Fecha} & \textbf{Autores}     & \textbf{Descripción}                                                 \\
    \hline
    1.0              & 01/02/2023     & Alejandro Barrachina & Apuntes en formato libro basados en las transparencias de Sara Román \\
    \hline
\end{tabularx}

\newpage

\chapterA{Introducción a la Ética y la Legislación}

\section{Introducción a la Ética}
La ética es una parte de la filosofía, la cual implica cuestionarse sobre problemas relativos al bien, al deber, a la virtud o al vicio.
Los objetivos de la ética son:
\begin{enumerate}
    \item Llegar a alguna conclusión acerca de lo que es correcto o no
    \item Que la conclusión pueda ser defendida con argumentos.
    \item Justificar las normas que regulan el comportamiento en diferentes ámbitos de la vida
\end{enumerate}


La reflexión es la consideración minuciosa de un asunto, esto implica dedicarle tiempo y no quedarse con la primera impresión. Para llevar a cabo esta consideración minuciosa debemos:
\begin{itemize}
    \item Informarnos: contrastar fuentes y poder citarlas.
    \item Pros y contrastar.
    \item Partes afectadas.
    \item Circunstancias.
    \item Posibles escenarios futuros.
    \item Contrastar: comentar, dialogar, escuchar otros puntos de vista diferentes al propio.
    \item Discernir, analizar
\end{itemize}

Es importante hacerse muchas preguntas para poder reconocer y tener en cuenta a todas las partes implicadas, dirigir esfuerzos a la hora de buscar información, ampliar nuestra visión del tema y compartir o hacer esas preguntas a otras personas para ampliar nuestra perspectiva.
Los elementos claves para una reflexión ética son:
\begin{enumerate}
    \item Identificar los valores del conflicto.
    \item Empatía: personas/colectivos afectados: ponerse en el lugar de los otros, percibir a los demás como seres sintientes.
    \item Analizar las circunstancias.
    \item Pros, contras y su peso.
    \item Alcance de las consecuencias en el espacio y el tiempo.
    \item Posibles escenarios futuros partiendo del presente.
\end{enumerate}

Tras una reflexión ética debemos llegar a una conclusión clara, directamente relacionada con los argumentos y ejemplos reales utilizados. Esta conclusión debe ser coherente con los argumentos y ejemplos usados. Si esta conclusión está matizada demostrará un correcto proceso de reflexión.

\section{Introducción a la Legislación}
% El derecho es el arte de lo bueno y lo justo 
El derecho es la técnica de dar a cada persona lo que le corresponde. También es un conjunto de normas organizadas, escritas o no, cuyo cumplimiento es obligatorio y pueden ser impuestas de forma coactiva, que sirven para asegurar la pacífica convivencia.

El Código Civil en su art. 1.1 señala como fuentes del Derecho:
\begin{enumerate}
    \item La Ley: norma vigente.
    \item La costumbre: conductas repetidas desde tiempos inmemoriales. Solo rige en defecto de ley aplicable y siempre que no contradiga la moral o al orden público y tiene que ser probada.
    \item Los \gls{pgd}: principios fundamentales. Se aplican en defecto de la Ley o costumbre. Además, su contenido está siempre presente en el ordenamiento jurídico.
\end{enumerate}

Estas fuentes de derecho tienen una jerarquía, lo que hace que una disposición no carezca de valor si contradice a una de rango superior. Ley $>$ costumbre $>$ \gls{pgd}.
La legislación refleja una serie de valores (Derechos, protección) y una serie de conflictos (delitos, sanción, penas).
\section{Cuestiones éticas relacionadas con el diseño de aplicaciones informáticas}
Tanto el diseño como el software son políticos, ya que podemos considerar que ``Los algoritmos son opiniones incrustadas en código''.

El diseño de un software no es neutro, ya que se tienen que considerar distintos puntos al hacer el programa:
\begin{itemize}
    \item Qué función o funciones realiza.
    \item Qué restricciones ponemos al usuario y cuales no.
    \item Qué datos vamos a recolectar y/o analizar.
    \item Qué podemos deducir sobre los usuarios.
    \item Qué mecanismos se van a utilizar para crear o no adicción entre los usuarios.
\end{itemize}

Un ejemplo de software político puede ser BOSCO, una aplicación financiada por el Gobierno Español utilizada por las compañías eléctricas para decidir quién tiene derecho a un descuento en la factura de la luz.

Otro ejemplo sería el sistema VeriPol utilizado por la Policía Nacional para detectar denuncias falsas.

\subsection{Manipulación online}
Algunos de los casos más conocidos son: Facebook con su publicidad dirigida a adolescentes estresados y deprimidos, Uber con estrategias para conseguir que sus conductores trabajen más horas o en determinadas zonas y Cambrige Analytica y sus mensajes políticos personalizados basados en perfiles de Facebook.

Dentro de la manipulación online podemos ver varios tipos:
\begin{itemize}
    \item\textbf{Persuasión (influencia):} cualquier forma de influencia e influencia a través de la discusión racional.
    \item\textbf{Persuasión en sentido explicito:} es visible, consciente y resistible (hay alternativas).
    \item\textbf{Manipulación:} es oculta, inconsciente y explota vulnerabilidades cognitivas emocionales y estructurales o individuales. La víctima no es consciente de que ha sido dirigida a una decisión determinada. Hay un beneficio para la parte manipuladora.
    \item\textbf{Manipulación + \gls{tic}:} hay grandes cantidades de información personal y algoritmos que analizan estos datos que son invisibles y omnipresentes. Generan campañas individualizadas con anuncios y mensajes a medida.
\end{itemize}

Estas acciones pueden producir daños materiales como gastar dinero en cosas que no necesitamos o gastar más de lo que queríamos o daños en cuanto a violación de la autonomía, con impacto individual y/o social y político.

Al estudiar un programa tenemos que hacernos las siguientes preguntas:
\begin{itemize}
    \item Quién decide qué y cómo se diseñan.
    \item Si tiene  mayor o menor potencial manipulador.
    \item Si está diseñada para ser adictiva.
    \item Si está diseñada para potenciar el comportamiento impulsivo y disminuir la capacidad analítica.
    \item Si se da al usuario una falsa sensación de autonomía.
\end{itemize}

\subsection{El salto a la \gls{ia}}

La evolución de la \gls{ia} nos ha traído sistemas muy complejos con aprendizajes profundos (encuentran patrones en un conjunto de datos) que sacan conclusiones estadísticas y son fáciles de engañar.

Al ser entrenadas mediante conjuntos de datos, ¿Pueden sus diseñadores predecir su comportamiento?  Algunas arquitecturas de \gls{ia} funcionan como una caja negra en la que no sabemos el porqué de las decisiones de la máquina, aunque no todas las arquitecturas son tan opacas.

Hay que tener en cuenta el hecho de que el sesgo cognitivo y los prejuicios de los desarrolladores de una \gls{ia} pueden influir de manera negativa en su toma de decisiones.
Estos sesgos pueden ser inconscientes o heredados(por el diseño o por el conjunto de datos que se usa para el entrenamiento) o prejuicios deliberados que nazcan de una determinada política o idea.

Para evitar \gls{ia}s con este tipo de problema tenemos que buscar siempre que el algoritmo sea explicable.

\epigraph{Cómo asegurar que el algoritmo es justo, cómo asegurar que el algoritmo es interpretable y explicable: todo eso está todavía bastante lejos.}{\textit{Nihar Shah}}

Es de interés para las personas que desarrollan el algoritmo, para abogades y jueces y para toda la ciudadanía entender la toma de decisiones de un algoritmo.

Parte de evitar problemas de bias viene en forma de auditorías, para las cuales hay que tener acceso a:
\begin{itemize}
    \item Diseño: explicabilidad, trazabilidad y reproducibilidad.
    \item Datos de entrenamiento o conjunto de datos para probar el sistema.
    \item Código fuente
\end{itemize}

\subsection{Guía ética para una \gls{ia} confiable (\gls{ue})}

\epigraph{To everyone who shapes technology today

    We live in a world where technology is consuming society, ethics, and our core existence.

    It is time to take responsibility for the world we are creating. Time to put humans before business. Time to replace the empty rhetoric of “building a better world” with a commitment to real action. It is time to organize, and to hold each other accountable.
}{\textit{The Copenhagen Letter, 2017}}

La \gls{ue} define una \gls{ia} fiable como una que es lícita, ética y robusta.
Una \gls{ia} fiable debe respetar cuatro principios éticos (a parte de respetar los derechos fundamentales):
\begin{itemize}
    \item Respeto de la autonomía humana.
    \item Prevención del daño.
    \item Equidad.
    \item Explicabilidad.
\end{itemize}

Una \gls{ia} fiable debe cumplir también siete requisitos clave:
\begin{itemize}
    \item Acción y supervisión humanas.
    \item Solidez técnica y seguridad.
    \item Transparencia.
    \item Diversidad, no discriminación y equidad.
    \item Bienestar social y ambiental.
    \item Rendición de cuentas.
\end{itemize}

Para aplicar estos principios hay que hacerlo desde el propio diseño del algoritmo: poniendo el respeto al ser humano por encima de cualquier otro tipo de interés como centro del diseño, respetar la autonomía individual y colectiva, no discriminar, revisar los sesgos, hacer un algoritmo explicable y transparente, respetar la privacidad y los \gls{ddaa} (licencia y opciones de protección de los \gls{ddaa}) y prever y tratar de evitar posibles usos fraudulentos y/o delictivos del algoritmo.

\chapterA{Privacidad}

\section{Derecho fundamental}

La privacidad es un derecho fundamental recogido en la Constitución Española de 1978 en el Articulo 18:
\begin{enumerate}
    \item Se garantiza el derecho al honor, a la intimidad personal y familiar y a la propia imagen.
    \item El domicilio es inviolable. Ninguna entrada o registro podrá hacerse en él sin consentimiento del titular o resolución judicial, salvo en caso de flagrante delito.
    \item Se garantiza el secreto de las comunicaciones y, en especial, de las postales, telegráficas y telefónicas, salvo resolución judicial.
    \item La ley limitará el uso de la informática para garantizar el honor y la intimidad personal y familiar de los ciudadanos y el pleno ejercicio de sus derechos.
\end{enumerate}

\section{Redes sociales}

Las redes sociales tienen diseños basados en crear adicción (modelo ``Hooked'' de Mil Eyal), con técnicas de persuasión/ manipulación / coacción y extracción de datos del usuario para su monetización.

Los términos y condiciones de una red social recogen datos importantes como:
\begin{itemize}
    \item Copyright
    \item Información que recogen y cómo la almacenan:
          \begin{itemize}
              \item Qué información.
              \item Donde (transferencia de datos fuera de la \gls{ue}).
              \item Durante cuánto tiempo.
              \item Con quién la comparten.
              \item Para qué la utilizan.
          \end{itemize}
    \item Cambios, notificaciones.
    \item Como cerrar una cuenta y que pasa con ella después de cerrarla,
    \item Cookies de rastreo
    \item Censura de contenido
\end{itemize}

Pero normalmente la gente no las lee, ya que son largas y complicadas con mucho texto jurídico. Es importante leer estos términos de servicio para ser conscientes de lo que hace la aplicación con nuestros datos.

Es importante también saber que una clausula o termino ilegal no tiene validez aunque aparezca en los términos de servicio y que toda empresa que opere en la \gls{ue} debe cumplir con el \gls{rgpd}

El auge de las redes sociales a traído también el ciberacoso, que tiene un mayor impacto y difusión que el acoso ``tradicional''. Podemos decir que hay tres tipos de ciberacoso: exclusión, manipulación y hostigamiento, también podemos diferenciar tres partes: persona que acosa, persona que es acosada y persona observadora.

El ciberacoso puede causar una reacción en cadena, ocasionando un efecto de bola de nieve. No hay violencia física pero causa un perjuicio muy grande en la víctima.

Al usar las redes sociales es importante pensar en qué uso damos de ellas, cuánto las usamos y si las usamos de manera correcta. Es importante el impacto de nuestros actos en las redes sociales.
% \section{Filtraciones}
% \section{Vigilancia}
% \section{Criptografía}
\section{\gls{lopd} y \gls{rgpd}}

El \gls{rgpd} se define en el Reglamento (\gls{ue}) \textbf{2026/679} del Parlamento Europeo y del Consejo de 27 de abril de 2016 y es relativo a la protección de las personas físicas en lo que respecta al tratamiento de datos personales y a la libre circulación de estos datos y por el que se deroga la Directiva 95/46/CE.

La \gls{lopd} es una adaptación de este reglamento en nuestro país, cada país miembro tiene su propia adaptación.


El \gls{rgpd} define su entrada en vigor y aplicación en el \textbf{Artículo 99}:
\begin{itemize}
    \item El presente Reglamento entrará en vigor a los veinte días de su publicación en el Diario Oficial de la Unión Europea.
    \item Será aplicable a partir del 25 de mayo de 2018.
\end{itemize}

El presente reglamento será obligatorio en todos sus elementos y directamente aplicable en cada Estado miembro.

El esquema del \gls{rgpd} es:
\begin{enumerate}
    \item 173 consideraciones (28 páginas): por qué, para qué (preámbulo).
    \item Disposiciones Generales.
    \item Ámbito de aplicación: actividades realizadas en la \gls{ue} independientemente de donde realices el tratamiento.
    \item Definiciones
    \item Principios: relativos al tratamiento, al consentimiento, definir categorías especiales de datos personales etc.
    \item Derechos: transparencia, información y acceso, rectificación, oposición + art. 17 $\rightarrow$ Derecho al olvido.
    \item Limitaciones: casos judiciales etc.
    \item Obligaciones de responsables del tratamiento.
\end{enumerate}

\chapterA{Derechos Digitales}

Los avances digitales tales como la \gls{ia}, realidad virtual y aumentada o la robótica suponen nuevos retos para la legislación:
\begin{itemize}
    \item Protección de datos personales.
    \item Procesamiento de \textit{big data} justo y respetuoso.
    \item Internet de las cosas.
    \item Robótica.
    \item Sistemas de \gls{ia}.
\end{itemize}

El derecho al respeto por la vida privada es una preocupación, ya que muchas aplicaciones \gls{tic} intentan influir en las actitudes y comportamientos de las personas.
Dichas actividades persuasivas afectan a la autonomía de la persona, a su capacidad de autodeterminación y a si libertad de pensamiento y de conciencia.

Los derechos digitales surgen por las limitaciones de los derechos fundamentales ``tradicionales'' en el ámbito digital, son medidas necesarias para garantizar el respeto de los derechos fundamentales en este ámbito.
Algunos de estos derechos son el derecho a que ``me dejen en paz'', el derecho al olvido y el derecho al acceso a Internet(no es lo mismo que acceso gratuito) entre otros.

\section{Libertad de expresión}

La \textbf{libertad de expresión} se recoge en el Artículo 20 de la constitución y es un derecho fundamental.
La libertad de expresión en Internet se mantiene igual, pero con un carácter más global. Cualquier persona puede hacer públicas sus opiniones, puntos de vista etc. con una mayor posibilidad de impacto y difusión que por otros medios.
Se mantienen los mismos límites establecidos por la constitución:
\begin{itemize}
    \item Respeto al honor.
    \item Respeto a la intimidad.
    \item Respeto a la propia imagen.
    \item Respeto a la juventud y a la infancia.
\end{itemize}

En internet hay filtros en las distintas para plataformas para proteger el copyright, aunque podrían utilizarse para retirar contenidos que ``no interesen''. Son muchos los ejemplos de plataformas que usan estos filtros para imponer su ideología y estándares.

El \textbf{anonimato} garantiza poder expresar opiniones sin temor a las represalias. En internet garantiza la libertad de expresión y también el libre intercambio de información y la privacidad y derecho a no ser espiados. El anonimato se declara como parte del derecho a la libertad de expresión (art. 20 \gls{ce}) y como parte del secreto de las comunicaciones (art 18 \gls{ce})
Así, la extensión del secreto de las comunicaciones a las comunicaciones electrónicas, la garantía de un cierto derecho al anonimato cuando se navegue por Internet, se hagan transacciones económicas o se participe políticamente a través de la Red, aparece como uno de los más importantes derechos, a la vez que más discutido, en la actualidad.

La suma de la IP, mas las cookies, más la minería de datos, puede resultar en la identificación de una persona. Para identificar a una persona(por ejemplo al investigar un delito cometido desde una IP) es necesaria la IP y ``medios que pueden ser razonablemente utilizados'' para asegurar la correspondencia IP-persona.
La protección otorgada a las direcciones IP constituye, por lo tanto, un elemento esencial para mantener el anonimato en Internet. Grandes empresas de Internet (como Google) han cuestionado que la IP sea un dato de carácter personal. Una misma IP puede ser compartida por diferentes usuarios de un mismo \gls{isp} (IPs dinámicas).

\section{Transparencia}

La relación entre publicidad y privacidad o los derechos de acceso a la información, a la intimidad y a la protección de datos es potencialmente conflictiva.
Convergen en un punto de conexión, la divulgación por las autoridades públicas de información que contienen datos personales, lo que quiere dilucidar cuál es la normativa aplicable y las determinaciones sustantivas, procedimentales, de garantías y organizativas que permitan maximizar la eficacia de ambos derechos. Y, a demás, hacerlo de forma adaptar al mundo digital en que actualmente vivimos.

Sobre esta problemática particular se presenta el conflicto entre publicidad y privacidad de la información pública en Internet, y a falta de Autoridades de transparencia y acceso a la información, el protagonismo lo está ejerciendo la \gls{aepd}, que ha dictado resoluciones y recomendaciones del mayor interés sobre esta materia.

\section{Neutralidad de la red}

Todos los paquetes que viajan por la red deben recibir el mismo tratamiento por parte de los \gls{isp} y los gobiernos, no se privilegia a ningún participante por encima de otro.
No se debe cobrar diferente en función del contenido al que se acceda, plataforma, aplicación o tipo de equipamiento utilizado para el acceso.
Esto es muy importante porque garantiza la igualdad de acceso a contenidos de Internet y porque garantiza la privacidad de la información que viaja por la red (que tendría que ser examinada para ser tratada de diferente forma).

\section{Criptografía: derecho fundamental}

Para mantener la privacidad y el anonimato es importante encriptar ciertos tipos de información:
\begin{itemize}
    \item Comunicaciones personales.
    \item Transacciones monetarias.
    \item Contraseñas, números de tarjetas de crédito etc.
    \item Información empresarial.
\end{itemize}

Hay un debate sobre si los gobiernos deberían tener acceso a datos encriptados (Estados Unidos contra Apple, Rusia contra Telegram).

Las leyes sobre criptografía tienen algunas restricciones:
\begin{itemize}
    \item\textbf{Control de exportaciones:} que es la restricción a exportar métodos de criptografía desde un país a otro país o entidad comercial. Hay acuerdos de exportación internacionales, siendo el principal el Acuerdo de Wassenaar.
    \item\textbf{Control de importaciones:} este punto se refiere a las restricciones de usar ciertos métodos de encriptado en un país.
    \item\textbf{Problemas con patentes.}
    \item En algunas ocasiones una persona puede ser obligada a desencriptar archivos o revelar una clave de encriptado.
\end{itemize}

En \gls{eeuu} es necesario pedir permiso antes de publicar un algoritmo o software de cifrado y tienen una regulación de algoritmos criptográficos fuera de su país, el \gls{ear} parte del International Traffic in Arms Regulation.

\section{Comunidades online / virtuales}

Se denomina comunidad virtual a aquella cuyos vínculos, interacciones y relaciones tienen lugar, no en un espacio físico sino en un espacio como Internet.
Las comunidades online se forman a partir de intereses similares entre grupos de personas. Se organizan y se llevan a cabo a partir de objetivos específicos.
Las comunidades saben que son redes, evolucionan de este modo, ampliando los miembros, diversificándose entre sí, nacen en el ciberespacio.

Podemos ver comunidades centralizadas y distribuidas, gobernadas por Empresas, gobiernos o autogestionadas.

\chapterA{Brecha Digital y Privilegios}
\section{Privilegios y desigualdad}

\epigraph{Los españoles son iguales ante la ley, sin que pueda prevalecer discriminación alguna por razón de nacimiento, raza, sexo, religión, opinión o cualquier otra condición o circunstancia personal o social}{\textit{Artículo 14 \gls{ce}}}

En la sociedad podemos ver una igualdad de derechos, pero una desigualdad de oportunidades.

Debemos ver los privilegios de dos maneras:
\begin{itemize}
    \item Con relación a la ética: con la capacidad de tener en cuenta a todas las partes implicadas en un conflicto y ponerse en su lugar.
    \item En relación con la brecha digital: plantearse si el acceso a la tecnología es o no un privilegio y si tendría que considerarse un Derecho Fundamental.
\end{itemize}

\section{Brecha Digital}

El \gls{desi} es un informe anual publicado por la Comisión Europea que supervisa los avances de los Estados Miembros de la \gls{ue} en el ámbito digital. En este ranking España supera la media europea de personas con capacidades digitales básicas ($64\%$ frente a un $54\%$), pero está por debajo de la media en la proporción de especialistas y titulados en \gls{tic}.

Se consideran competencias digitales básicas saber que existe el correo electrónico y nivel medio al saber utilizarlo.


En España, según datos del \gls{ine}, hay un $1,6\%$ de personas sin habilidades digitales, un $31,4\%$ con habilidad baja y un $19,1\%$ con habilidades básicas, lo que implica que más de la mitad de la población no sepa usar el correo electrónico.

Tenemos que plantearnos si es un verdadero avance estar más conectados sin resolver los problemas éticos relacionados con el diseño como con el uso de la tecnología digital, como si es ético a cada vez obligar a realizar más trámites administrativos por internet.

\section{Brecha de género}

Podemos ver esta brecha de género en las \gls{tic} extrapolando datos de nuestra propia facultad:
\begin{itemize}
    \item\textbf{Curso 2018/2019}:
    \begin{itemize}
        \item $15,3\%$ mujeres(total)
        \item Nuevos ingresos:  $14\%$
    \end{itemize}
    \item\textbf{Curso 2020/2021}:
    \begin{itemize}
        \item $19,4\%$ mujeres(total)
        \item Nuevos ingresos:  n/s
    \end{itemize}
    \item\textbf{Curso 2021/2022}:
    \begin{itemize}
        \item $19,3\%$ mujeres(total)
        \item Nuevos ingresos:  $18\%$
    \end{itemize}
\end{itemize}

Algunas de las posibles causas de esta brecha cuantitativa son:
\begin{itemize}
    \item Falta de referentes femeninos en la profesión.
    \item Percepción de menor capacidad de las mujeres.
    \item Falta de interés natural de las mujeres.
    \item Programas educativos / diseño de entorno muy masculino que no resulta atrayente.
    \item Estereotipos.
    \item Publicidad relacionada con la informática dirigida a público masculino.
\end{itemize}

Las primeras programadoras eran mujeres(``Top secret rosies'':Kathleen
McNulty Mauchly, Marlyn Wescoff Meltzer, Betty Snyder Holberton, Jean
Jennings Bartik, Frances Bilas Spencer y Ruth Lichterman Teitelbaum ), los primeros avances en el software los realizaron mujeres: el primer compilador, primeros lenguajes de alto nivel, primer procesador de texto y el propio termino ``bug''.

A finales de los años 60 y principios de los 70 el software comienza a tener un valor económico y se empiezan a cotizar más los puestos de programación ya que estaban mejor pagados. Esto llevó a que la presencia de las mujeres en la profesión disminuyese al $25\%$.

La falta de presencia de mujeres en las \gls{tic}  tiene como consecuencia que un $51\%$ de la población esté infra-representada en el diseño de la sociedad actual y futura. La falta de diversidad conlleva una peor calidad, menor innovación, menos ingresos y diseños que reproducen estereotipos y discriminaciones, manteniendo así los sesgos.

En 1997 un estudio demuestra que en Suecia una mujer necesita hasta 2,4 veces más méritos que un hombre para recibir una beca pos-doctoral (\url{https://www.nature.com/articles/387341a0}). En 2012 un experimento demuestra que el mismo currículum atribuido a un hombre recibe mayor valoración que cuando es atribuido a una mujer  (\url{http://www.pnas.org/content/early/2012/09/14/1211286109})

Estos son solo dos ejemplos de decenas que demuestran que en las profesiones relacionadas con las ciencias hay una clara preferencia hacia los hombres, dejando en una injusta desventaja a las mujeres.

También son muchos los casos de empresas que han sido llevadas a juicio por pagar menos a empleadas frente a sus compañeros masculinos con el mismo puesto:
\begin{itemize}
    \item \url{https://eu.usatoday.com/story/tech/2017/09/29/oracle-yet-another-tech-firm-hit-suit-allegedly-paying-women-less-than-men/718471001/}
    \item \url{https://eu.usatoday.com/story/tech/2017/09/14/google-hit-gender-pay-gap-lawsuit-seeking-class-action-status/666944001/}
    \item \url{https://www.mercurynews.com/2019/09/19/google-paid-female-engineering-director-less-demoted-her-for-complaining-gender-discrimination-lawsuit/}
\end{itemize}

\chapterA{Derechos de autor}

\section{Propiedad intelectual, derechos de autor, \gls{lpi}}

\subsection{Productos objeto y derechos del autor}

El Artículo 10 de la \gls{lpi} define que productos son objeto de propiedad intelectual:
\begin{enumerate}[label=\textbf{\arabic*.}]
    \item Son objeto de propiedad intelectual todas las creaciones originales literarias, artísticas o científicas expresadas por cualquier medio o soporte, tangible o intangible, actualmente conocido o que se invente en el futuro, comprendiéndose entre ellas:
          \begin{enumerate}[label=\textbf{\alph*)}]
              \item Los libros, folletos, impresos, epistolarios, escritos, discursos, alocuciones, conferencias, informes forenses, explicaciones de cátedra y cualesquiera otras obras de la misma naturaleza.
              \item Las composiciones musicales, con o sin letra.
              \item Las obras dramáticas y dramático-musicales, las coreografías, las pantomimas y, en general, las obras teatrales.
              \item Las obras cinematográficas y cualesquiera otras obras audiovisuales.
              \item Las esculturas y las obras de pintura, dibujo, grabado, litografía y las historietas gráficas, tebeos o cómics, así como sus ensayos o bocetos y las demás obras plásticas, sean o no aplicadas.
              \item Los proyectos, planos, maquetas y diseños de obras arquitectónicas y de ingeniería.
              \item Los gráficos, mapas, diseños relativos a la topografía, la geografía y, en general, a la ciencia.
              \item Las obras fotográficas y las expresadas por procedimiento análogo a la fotografía.
              \item Los programas de ordenador
          \end{enumerate}
    \item El titulo de una obra, cuando sea original, quedará protegido como parte de ella.
\end{enumerate}


El Capítulo III de la \gls{lpi} define los derechos morales del autor (resumen):

Pertenecen al autor, no pueden cederse:
\begin{itemize}
    \item Decidir si su obra ha de ser divulgada y en qué forma.
    \item Exigir el reconocimiento de su condición de autor de la obra.
    \item Retirar la obra del comercio, por cambio de sus convicciones intelectuales o morales, previa indemnización de daños y prejuicios a los titulares de derechos de explotación.
\end{itemize}

Pueden cederse:
\begin{itemize}
    \item Obtención de beneficios
    \item Reproducción, distribución, comunicación pública y transformación.
\end{itemize}


\subsection{Dominio público}

En los Artículos 26 y 30 se define el tiempo que una obra está amparada bajo la \gls{lpi}:

\textbf{Artículo 26: Duración y cómputo}\\
Los derechos de explotación de la obra durarán toda la vida del autor y setenta años después de su muerte o declaración de fallecimiento.

\textbf{Artículo 30: Cómputo del plazo de protección}\\
Los plazos de protección establecidos en esta Ley se computarán desde el día 1 de enero del año siguiente al de la muerte o declaración de fallecimiento del autor o a la de la divulgación lícita de la obra, según proceda.

Pasados los 70 años las obras pasan al dominio público.

\subsection{Límites y excepciones}

Algunos artículos ponen excepciones y límites a los derechos del autor:
\begin{itemize}
    \item\textbf{Artículo 31} Reproducciones provisionales y copia privada.
    \item\textbf{Artículo 31 bis} Seguridad, procedimientos oficiales y discapacidades.
    \item\textbf{Artículo 32} Cita e ilustración de la enseñanza.
    \item\textbf{Artículo 33} Trabajos sobre temas de actualidad.
    \item\textbf{Artículo 34} Utilización de bases de datos por el usuario legítimo y limitaciones a los derechos de explotación del titular de una base de datos.
    \item\textbf{Artículo 35} Utilización de las obras con ocasión de informaciones de actualidad y de las situadas en vías públicas.
    \item\textbf{Artículo 36} Cable, satélite y grabaciones técnicas.
    \item\textbf{Artículo 37} Reproducción, préstamo y consulta de obras mediante terminales especializados en determinados establecimientos.
    \item\textbf{Artículo 38} Actos oficiales y ceremonias religiosas.
    \item\textbf{Artículo 39} Parodia.
\end{itemize}

\subsection{Sobre la copia privada}

\textbf{Artículo 31. Reproducciones provisionales y copia privada}
\begin{enumerate}[label=\textbf{\arabic*.}]
    \item No requerirán autorización del autor los actos de reproducción provisional a los que se refiere el artículo 18 que, además de carecer por si mismos de una significación económica independiente, sean transitorios o accesorios y formen parte integrante y esencial de un proceso tecnológico y cuya única finalidad consista en facilitar bien una transmisión en red entre terceras partes por un intermediario, bien una autorización lícita, entendiendo por tal la autorizada por el autor o por la ley.
    \item Sin perjuicio de la compensación equitativa prevista en el artículo 25, no necesita autorización del autor la reproducción, en cualquier soporte, sin asistencia de terceros, de obras ya divulgadas cuando ocurran simultáneamente las siguientes circunstancias, constitutivas del límite legal de la copia privada:
          \begin{enumerate}[label=\textbf{\alph*)}]
              \item Que se lleve a cabo por una persona física exclusivamente para su uso privado, no profesional ni empresarial, y sin fines directa ni indirectamente comerciales.
              \item Que la reproducción se realice a partir de una fuente lícita y que no se vulnere las condiciones de acceso a la obra o presentación.
              \item Que la copia obtenida no sea objeto de una utilización colectiva ni lucrativa, ni de distribución mediante precio.
          \end{enumerate}
    \item Quedan excluidas de lo dispuesto en el anterior apartado:
          \begin{enumerate}[label=\textbf{\alph*)}]
              \item Las reproducciones de obras que se hayan puesto a disposición del público conforme al artículo 20.2.i), de tal forma que cualquier persona pueda acceder a ellas desde el lugar y momento que elija autorizándose, con arreglo a lo convenido por contrato y, en su caso, mediante pago de precio, la reproducción de la obra.
              \item Las bases de datos electrónicas.
              \item Los programas de ordenador, en aplicación de la letra a) del artículo 99.
          \end{enumerate}
\end{enumerate}

\textbf{Artículo 25. Compensación equitativa por copia privada}
\begin{enumerate}[label=\textbf{\arabic*.}]
    \item Las reproducciones de obras divulgadas en forma de libros o publicaciones que a estos efectos se asimilen mediante real decreto, así como de fonogramas, videogramas o de otros soportes sonoros, visuales o audiovisuales, realizada mediante aparatos o instrumentos técnicos no tipográficos, exclusivamente para uso privado, no profesional ni empresarial, sin fines directa o indirectamente comerciales, de conformidad con el artículo 31, apartados 2 y 3, originará una compensación equitativa y única para cada una de las tres modalidades de reproducción mencionadas dirigidas a compensar adecuadamente el perjuicio causado a los sujetos acreedores como consecuencia de las reproducciones realizadas al amparo del límite legal de copia privada.
    \item Serán sujetos acreedores de esta compensación equitativa y única los autores de las obras señaladas en el apartado anterior, explotadas públicamente en alguna de las formas mencionadas en dicho apartado, conjuntamente y, en los casos y modalidades de reproducción en que corresponda, con los editores, los productores de fonogramas y videogramas y los artistas intérpretes o ejecutantes cuyas actuaciones hayan sido fijadas en dichos fonogramas y videogramas. Este derecho será irrenunciable para los autores y los artistas intérpretes o ejecutantes.
\end{enumerate}


\section{Protección legal del software}

\subsection{Referencia histórica}

Se proponen dos vías para la protección del software, o bien por patente de invención o bien por derechos de autor. Se optó finalmente por derechos de autor dado que las patentes no ofrecían la suficiente protección para el software. Un software no puede reunir información suficiente para conocer el ``estado de la técnica'' del mismo, imposibilitando el hacer el examen de novedad y actividad inventiva. La Propiedad Intelectual ofrece protección con un mínimo de formalidades y costos y es la más adecuada para la cantidad de software que se genera.

Un programa de ordenador sería patentable como componente de un procedimiento de fabricación o de un aparato protegido por patente o por modelo de utilidad, y exclusivamente para la aplicación del mismo.

\section{?`Qué se puede proteger?}

Se pueden proteger programas como sistemas operativos, controladores y utilidades, compiladores, bibliotecas y entornos de desarrollo y utilidades, guiones y procedimientos almacenados, servidores web y de aplicaciones y código empotrado, firmware y microcódigo.

También se puede proteger la documentación de dichos programas: documentos de análisis de requisitos y de diseño del sistema, planes de pruebas, manuales de instalación y usuario, manuales de referencia, y ayuda interactiva.

No se puede proteger ni ideas, ni algoritmos, ni fórmulas matemáticas, ni principios generales ni interfaces de usuario o de aplicación.

\section{?`A quién pertenece el software?}

\textbf{Artículo 96. Objeto de la protección}

\begin{itemize}
    \item[\textbf{1.}] A los efectos de la presente Ley se entenderá por programa de ordenador toda secuencia de instrucciones o indicaciones destinadas a ser utilizadas, directa o indirectamente, en un sistema informático para realizar una función o una tarea o para obtener un resultado determinado, cualquiera que fuere su forma de expresión y fijación.\\
        A los mismos efectos, la expresión programas de ordenador comprenderá también su documentación preparatoria. La documentación técnica y los manuales de uso de un programa gozaran de la misma protección que este Título dispensa a los programas de ordenador.

    \item[\textbf{4.}] No estarán protegidos mediante los derechos de autor con arreglo a la presente Ley las ideas y principios en los que se basan cualquiera de los elementos de un programa de ordenador, incluidos los que sirven de fundamento a sus interfaces.

\end{itemize}

\textbf{Artículo 97. Titularidad de los derechos}
\begin{enumerate}[label=\textbf{\arabic*.}]
    \item Será considerado autor del programa de ordenador la persona o grupo de personas naturales que lo hayan creado, o la persona jurídica que sea contemplada como titular de los derechos de autor en los casos expresamente previstos por esta Ley.
    \item Cuando se trate de una obra colectiva tendrá la consideración de autor, salvo pacto de lo contrario, la persona natural o jurídica que la edite y divulgue bajo su nombre.
    \item Los derechos de autor sobre un programa de ordenador que sea resultado unitario de la colaboración entre varios autores serán propiedad común y corresponderán a todos éstos en la proporción que determinen.
    \item Cuando un trabajador asalariado cree un programa de ordenador, en el ejercicio de las funciones que le han sido confiadas o siguiendo las instrucciones de su empresario, la titularidad de los derechos de explotación correspondientes al programa de ordenador así creado, tanto el programa fuente como el programa objeto, corresponderán, exclusivamente, al empresario, salvo pacto en contrario.
    \item La protección se concederá a todas las personas naturales y jurídicas que cumplan los requisitos establecidos en esta Ley para la protección de los derechos de autor.
\end{enumerate}


\section{Límites a los derechos de explotación}

\textbf{Artículo 100. Límites a los derechos de explotación}
\begin{itemize}
    \item[label=\textbf{1.}] No necesitarán autorización del titular, salvo disposición contractual en contrario, la reproducción o transformación de un programa de ordenador incluida la corrección de errores, cuando dichos actos sean necesarios para la utilización del mismo por parte del usuario legítimo, con arreglo a su finalidad propuesta.
    \item[label=\textbf{2.}] La realización de una copia de seguridad por parte de quien tiene derecho a utilizar el programa no podrá impedirse por contrato en cuanto resulte necesaria para dicha utilización.
    \item[label=\textbf{3.}] El usuario legítimo de la copia de un programa estará facultado para observar, estudiar o verificar su funcionamiento sin autorización previa del titular, con el fin de determinar las ideas y principios implícitos en cualquier elemento del programa, siempre que lo haga durante cualquiera de las operaciones de carga, visualización, ejecución, transmisión o almacenamiento del programa que tiene derecho a hacer.
    \item[label=\textbf{5.}] No será necesaria la autorización del titular del derecho cuando la reproducción del código y la traducción de su forma en el sentido de los párrafos a) y b)  del artículo 99 de la presente LEy, sea indispensable para obtener la información necesaria para la interoperabilidad de un programa creado de forma independiente con otros programas, siempre que se cumplan los siguientes requisitos:
        \begin{enumerate}[label=\textbf{\alph*)}]
            \item Que tales actos sean realizados por el usuario legítimo o por cualquier otra persona facultada para utilizar una copia del programa, o, en su nombre, por parte de una persona debidamente autorizada.
            \item Que la información necesaria para conseguir la interoperabilidad haya sido puesta previamente y de manera fácil y rápida, a disposición de las personas a que se refiere el párrafo anterior.
            \item Que dichos actos se limiten a aquellas partes del programa original que resulten necesarias para conseguir la interoperabilidad.
        \end{enumerate}
\end{itemize}

\section{Protección ``sui generis'' de las Bases de Datos}

\textbf{Artículo 12. Colecciones. Bases de datos.}

\begin{enumerate}[label=\textbf{\arabic*.}]
    \item También son objeto de propiedad intelectual, en los términos del libro I de la presente Ley, las colecciones de obras ajenas, de datos o de otros elementos independientes como las antologías y las bases de datos que por la sección o disposición de sus contenidos construyan creaciones intelectuales, sin perjuicio, en su caso, de los derechos que pudieran subsistir sobre dichos contenidos.\\
          La protección reconocida en el presente artículo a estas colecciones se refiere únicamente a su estructura en cuanto forma de expresión de la selección o disposición de sus contenidos, no siendo extensiva a estos.
    \item A efectos de la presente LEy, y sin perjuicio de lo dispuesto en el apartado anterior, se consideran bases de datos las colecciones de obras, de datos o de otros elementos independientemente dispuestos de manera sistemática o metódica y accesibles individualmente por medios electrónicos o de otra forma.
    \item La protección reconocida a las ases de datos en virtud del presente artículo no se aplicará a los programas de ordenador utilizados en la fabricación o en el funcionamiento de bases de datos accesibles por medios electrónicos.
\end{enumerate}

\section{Copyright}

Cuando una obra está protegida bajo copyright no está permitida la reproducción total o parcial de la misma, ni su tratamiento informático, ni la transmisión de ninguna forma o cualquier medio, ya sea electrónico, mecánico, por fotocopia, por registro u otros métodos, sin el permiso previo y por escrito de los titulares del copyright. Reserva también todos los derechos, incluido el derecho a venta, alquiler, préstamo o cualquier otra forma de cesión del uso del ejemplar.


\section{Patentes}

Una patente es un título que reconoce el derecho de explotar en exclusiva la invención patentada, impidiendo a otros su fabricación, venta o utilización sin consentimiento del titular. Como contrapartida, la patente se pone a disposición del público para generar conocimiento.

La patente puede referirse a un procedimiento nuevo, un apartado nuevo, un producto nuevo o un perfeccionamiento o mejora de los mismos. La duración de una patente es de veinte años a contar desde la fecha de presentación de la solicitud. Para mantenerla en vigor es preciso pagar tasas anuales a partir de su concesión.

\chapterA{Cultura libre}

\epigraph{Free software is a matter of liberty, not price. To understand the concept you should think of free as in free speech, not as in free beer}{Richard Stallman}

\section{Software libre, de código abierto y gratuito}

Software libre y software gratuito no son lo mismo. En ingles \textit{Free software} se refiere a software libre.

Según \url{gnu.org} las libertades del software son las siguientes:

\begin{itemize}
    \item[\textbf{0.}] La libertad de ejecutar el programa para cualquier propósito.
    \item[\textbf{1.}] La liberta de estudiar cómo funciona el programa , y cambiarlo para que haga lo que usted quiera. El acceso al código fuente es una condición necesaria para ello.
    \item[\textbf{2.}] La libertad de redistribuir copias para ayudar a su prójimo.
    \item[\textbf{3.}] La libertad de distribuir copias de sus versiones modificadas a terceros. Esto le permite ofrecer a toda la comunidad la oportunidad de beneficiarse de beneficiarse de las modificaciones. El acceso al código fuente es una condición necesaria para ello.
\end{itemize}

\textbf{Copyleft:} mantenimiento de las condiciones de la licencia en toda la cadena de distribución. No permite que legalmente se puedan cerrar creaciones puestas a disposición del público de forma libre.

\textbf{Código abierto:} no se exige la distribución libre del código modificado. Permite que creaciones puestas a disposición del público libre se puedan cerrar.

\section{Historia}

En 1938 empieza el proyecto GNU (GNU is Not Unix) y en 1985 se funda la \gls{fsf}. El movimiento de software libre es un movimiento ético y político, ya que se fundamenta en que tener el control de la tecnología que usamos para que trabaje para nosotros y no para corporaciones o gobiernos que busquen restringirnos y monitorearnos.

El concepto de software libre ya existía, a finales de los 60 había software gratuito y empezaba a aparecer software (licencias de uso) de pago y con restricciones de uso.

A principios de los 70 AT\&T distribuyó copias gratuitas de Unix y conforme se fue extendiendo su uso, a principios de los 80 empezó a cobrar por ellas. En los 80, en paralelo al desarrollo comercial del software existían comunidades que compartían software libre online.

El término Software de Código Abierto lo adoptó un grupo de dentro del movimiento de software libre en 1998 que querían desmarcarse de la posición más radical y poco comercial del término software libre.

Raymond fundó la Open Source Initiative en 1998 junto a otras personas, Stallman y más miembros de la \gls{fsf} se opusieron al término y concepto dividiendo el movimiento.

En recientes años han aparecido los llamados \textit{hacktivistas} como Aaron Swartz ``Internet's own boy' que luchó por la privacidad en internet hasta que le impusieron una multa astronómica y 35 años de prisión, terminando él por suicidarse.
\section{Licencias}

Algunas licencias populares son \gls{gpl} y \gls{bsd}.

Las características principales de \gls{gpl} son:
\begin{enumerate}
    \item Copia y distribución del código fuente original.
    \item Modificación.
    \item Distribución de las modificaciones, siempre que se hagan bajo la misma licencia y sin cobrar por ellas.
    \item Copia y distribución del ejecutable, siempre que se ponga a disposición el código fuente sin cobrar un extra por ello.
\end{enumerate}

Las características principales de \gls{bsd} son:
\begin{itemize}
    \item Uso, modificación, copia y redistribución sin restricción del código objeto o el fuente.
    \item Aviso de copyright, negación de cualquier garantía o responsabilidad y prohibición de usar el nombre del autor con fines de promoción de obras derivadas sin su permiso.
    \item No se otorga ninguna garantía sobre el producto ni se asume ninguna responsabilidad.
    \item Si el software es modificado, se puede distribuir bajo otro tipo de licencia y no es necesario proveer al usuario final del código fuente.
\end{itemize}


Otro conjunto de licencias es Creative Commons. Estas licencias tienen categorías:
\begin{itemize}
    \item\textbf{BY:} atribución
    \item\textbf{SA:} compartir igual (copyleft)
    \item\textbf{ND:} sin derivados
    \item\textbf{NC:} sin re-uso comercial
\end{itemize}

En una escala de libre a libre tendríamos:
\begin{itemize}
    \item[\textbf{Libre}]:
    \begin{itemize}
        \item CC0 (DP)
        \item CC-by (\gls{bsd})
        \item CC-by-sa (\gls{gpl}) (copyleft)
    \end{itemize}
    \item[\textbf{No libre}]
        \begin{itemize}
            \item CC by-nc-sa (copyleft)
            \item CC by-nd
            \item CC by-nc
            \item Derechos de autor (CC by-nc-nd)
        \end{itemize}
\end{itemize}


\section{Hardware libre}

Aplicando los mismos conceptos del software libre al hardware llegamos a que los usuarios debería poder distribuir copias del hardware, pero en el hardware no hay algo como las ``copias''.

Lo que si se puede liberar es el diseño del propio hardware, esto implica que el diseño debe cumplir las mismas libertades que el software libre. Entonces el hardware hecho con diseños libres se podrá considerar hardware libre.

Una de las ventajas del diseño libre de hardware es que varias empresas pueden hacer el mismo producto y así no depender de un solo distribuidor. Tener los diagramas de circuito o código en HDL nos permite estudiar el diseño para buscar errores en el diseño o funcionalidades maliciosas.

\gls{gpl} a partir de su versión 3 se diseña con el diseño libre de hardware en mente. Un circuito como topología no puede tener copyright (tampoco copyleft). Definiciones de circuitos escritas en HDL pueden tener copyright y por tanto copyleft, pero la topología que este código genera no.
Un dibujo de un circuito puede tener copyright (no voy a repetir lo que esto implica), pero solo cubre el dibujo o la distribución, pero no la topología. Cualquiera podría copiar esa misma topología de forma que se vea distinta, o escribir un código HDL distinto que produzca el mismo circuito.

\chapterA{Delitos Informáticos}

\section{Definición}

Hay una falta de acuerdo sobre la definición jurídica de delitos informáticos. No están reflejados como al en el Código Penal español, sino que se han añadido aspectos relacionados en delitos ya existentes y se ha añadido algún artículo relativo al daño causado a bienes informáticos.

Algunes juristas consideran necesario diferenciar intrusismo informático de delincuencia informática.
Para Esther Morón el intrusismo son``comportamientos de acceso o interferencia no autorizados, de forma subrepticia, a un sistema informático o red de comunicación electrónica de datos y utilización de los mismos sin autorización o más allá de lo autorizado''.

Los equipos informáticos son nuevos bienes jurídicos, se vela por la integridad de la información y del propio equipo, y a demás los bienes jurídicos que pueden ser accedidos y vulnerados por medios informáticos: patrimonio, intimidad, identidad, material con copyright, etc.

\section{Código Penal Español}

\textbf{Ley Orgánica 10/1995, de 23 de noviembre, del Código Penal}

\textbf{Exposición de Motivos}

Si se ha llegado a definir el organismo jurídico como conjunto de normal que regulan el uso de la fuerza, puede entenderse fácilmente la importancia del Código Penal en cualquier sociedad civilizada.

El código penal define los delitos y faltas que constituyen los presupuestos de la aplicación de la forma suprema que puede revestir el poder coactivo del Estado: la pena criminal. En consecuencia, ocupa un lugar preeminente en el conjunto del ordenamiento, hasta el punto que, no sin razón, se ha considerado como una especie de ``Constitución negativa''.

El Código Penal ha de tutelar los valores y principios básicos de la convivencia social.

\subsection{Tipos de delitos informáticos}

\textbf{Delitos informáticos ``puros''}
\begin{itemize}
    \item Delitos contra la intimidad: descubrimiento y revelación de secretos (art. 197, art. 197 bis, art. 197 ter).
    \item Delito de daños, con especial referencia al sabotaje informático (art. 264, art. 264 bis, art. 264 ter y art. 560).
    \item Tecnología destinada a la comisión de delitos: art. 400.
    \item De la consideración de terrorismo: art. 573.2.
    \item Utilización abusiva dde equipos terminales de comunicaciones: art. 256.
\end{itemize}

\textbf{Delitos ``tradicionales'' que se ven agravados por el uso de las \gls{tic}}:
\begin{itemize}
    \item Delitos relativos a la propiedad intelectual e industrial: art.270 y art. 278.
    \item Uso de las \gls{tic} para realizar estafas: art. 248
    \item Uso de las \gls{tic} en delitos relacionados con abuso a menores: art. 187 y art. 189.
    \item Uso de las \gls{tic} para amenazar: art. 169.
    \item Uso de las \gls{tic} para calumniar e injuriar: art. 205.
\end{itemize}

\subsection{TÍTULO X. Delitos contra la intimidad, el derecho a la propia imagen y a la inviolabilidad del domicilio: CAPÍTULO PRIMERO. Del descubrimiento y revelación de secretos.}

\textbf{Articulo 197}
\begin{enumerate}[label=\textbf{\arabic*.}]
    \item El qie, para descubrir los secretos o vulnerar la intimidad de otro, sin su consentimiento, se apodere de sus papeles, cartas, mensajes de correo electrónico o cualesquiera otros documentos o efectos personales, intercepte sus telecomunicaciones o utilice artificios técnicos de escucha, transmisión, grabación o reproducción del sonido o de la imagen, o cualquier otra señal de comunicación, será castigado con las penas de prisión de uno a cuatro años y multa de doce a veinticuatro meses.
    \item Las mismas penas se impondrán al que, sin estar autorizado, se apodere, utilice o modifique, en prejuicio de terceros, datos reservados de carácter personal o familiar de otro que se hallen registrados en ficheros o soportes informáticos, electrónicos o telemáticos, o en cualquier otro tipo de archivo o registro público o privado. Iguales penas se impondrán a quien, sin estar autorizado, acceda por cualquier medio a los mismos y a quien los altere o utilice en perjuicio del titular de los datos de un tercero.
    \item Se impondrá la pena de prisión de dos a cinco años si se difunden, revelan o ceden a terceros los datos o hechos descubiertos o las imágenes captadas a que se refieren los números anteriores.
\end{enumerate}

\textbf{Artículo 197 bis}
\begin{enumerate}[label=\textbf{\arabic*.}]
    \item El que por cualquier medio o procedimiento, vulnerando las medidas de seguridad establecidas para impedirlo y sin estar debidamente autorizado, acceda o facilite a otro el acceso al conjunto o una parte del sistema de información o se mantenga en él en contra de la voluntad de quien tenga el legítimo derecho a excluirlo, será castigado con pena de prisión de seis meses a dos años.
\end{enumerate}

\textbf{Artículo 197 ter}

Será castigado con una pena de prisión de seis meses a dos años o multa de tres a dieciocho meses el que, sin estar debidamente autorizado, produzca, adquiera para su uso, importe o, de cualquier modo, facilite a terceros, con la intención de facilitar la comisión de alguno de los delitos a que se refieren los apartados 1 y 2 del artículo 197 o el artículo 197 bis:
\begin{enumerate}[label=\textbf{\alph*)}]
    \item Un programa informático, concebido o adaptado principalmente para cometer dichos delitos; o
    \item Una contraseña de ordenador, un código de acceso o datos similares que permitan acceder a la totalidad o a una parte de un sistema de información.
\end{enumerate}

\subsection{TÍTULO XIII. Delitos contra el patrimonio y el orden socioeconómico: CAPÍTULO IX. De los daños}

\textbf{Artículo 264}
\begin{enumerate}[label=\textbf{\arabic*.}]
    \item El que por cualquier medio, sin autorización y de manera grave borrase, dañase, deteriorase, alterase, suprimiese o hiciese inaccesibles datos informáticos, programas informáticos o documentos electrónicos ajenos, cuando el resultado producido fuera grave, será castigado con la pena de prisión de seis meses a tres años.
    \item Se impondrá una pena de prisión de dos a cinco años y multa del tanto al décuplo del perjuicio ocasionado, cuando en las conductas descritas concurra alguna de las siguientes circunstancias:
          \begin{enumerate}[label=\arabic*.a]
              \item Se hubiese cometido en el marco de una organización criminal.
              \item Haya ocasionado daños de especial gravedad o afectado a un número elevado de sistemas informáticos.
              \item El hecho hubiera perjudicado gravemente el funcionamiento de servicios públicos esenciales o la provisión de bienes de primera necesidad.
              \item Los hechos hayan afectado al  sistema informático de una estructura crítica o se hubiera creado una situación de peligro grave para la seguridad del Estado, de la \gls{ue} o de un Estado Miembro de la \gls{ue}. A estos efectos se considerará infraestructura crítica a un elemento, sistema o parte de este que sea esencial para el mantenimiento de funciones vitales de la sociedad, la salud, la seguridad, la protección y el bienestar económico y social de la población cuya perturbación o destrucción tendría un impacto significativo al no poder mantener sus funciones.
          \end{enumerate}
          Si los hechos hubieran resultado de extrema gravedad, podrá imponerse la pena superior en grado.
    \item Las penas previstas en los apartados anteriores se impondrán, en sus respectivos casos, en su mitad superior, cuando los hechos se hubieran cometido mediante la utilización ilícita de datos personales de otra persona para facilitarse el acceso al sistema informático o para ganarse la confianza de terceros
\end{enumerate}

\textbf{Artículo 264 bis}

\begin{enumerate}[label=\textbf{\arabic*.}]
    \item Serán castigados con la pena de prisión de seis meses a tres años el que, sin estar autorizado y de manera grave,  obstaculizara o interrumpiera el funcionamiento de un sistema informático ajeno:
          \begin{enumerate}[label=\textbf{\alph*)}]
              \item Realizando alguna de las conductas a que se refiere el articulo anterior;
              \item Introduciendo o transmitiendo datos; o
              \item Destruyendo, dañando, inutilizando, eliminando o sustituyendo un sistema informático, telemático o de almacenamiento de información electrónica.
          \end{enumerate}
\end{enumerate}

\textbf{Artículo 264 ter}

Será castigado con una pena de seis meses a dos años o multa de tres a dieciocho meses el que, sin estar debidamente autorizado, produzca, adquiera para su uso, importe o, de cualquier modo, facilite a terceros, con la intención de facilitar la comisión de alguno de los delitos a que refiere los dos artículos anteriores:
\begin{enumerate}[label=\textbf{\alph*)}]
    \item Un programa informático, concebido o adaptado principalmente para cometer alguno de los delitos a que se refieren los dos artículos anteriores; o
    \item Una contraseña de ordenador, un código de acceso o datos similares que permitan acceder a la totalidad o a una parte de un sistema de información.
\end{enumerate}

\textbf{Artículo 573}
\begin{enumerate}[label=\textbf{\arabic*.}]
    \item Se considerará delito de terrorismo la comisión de cualquier delito grave contra la vida o la integridad física, la libertad, la integridad moral, la libertad e indemnidad sexuales, el patrimonio, los recursos naturales o el medio ambiente, la salud pública, de riesgo catastrófico, incendio, contra la Corona, de atentado y tenencia, trafico y depósito de armas, municiones o explosivos, previstos en el presente Código, y el apoderamiento de aeronaves, buques u otros medios de transporte colectivo o de mercancías, cuando se llevaran a cabo con cualquiera de las siguientes finalidades:
          \begin{enumerate}[label=\arabic*.a]
              \item Subvenir el orden constitucional, o suprimir o desestabilizar gravemente el funcionamiento de las instituciones políticas o de las estructuras económicas o sociales del Estado, u obligar a los poderes públicos a realizar un acto o a abstenerse de hacerlo.
              \item Alterar gravemente la paz pública.
              \item Desestabilizar gravemente el funcionamiento de una organización internacional.
              \item Provocar un estado de terror en la población o en una parte de ella.
          \end{enumerate}
    \item Se considerarán igualmente delitos de terrorismo los delitos informáticos tipificados en los artículos 197 bis y 197 ter y 264 a 264 quater cuando los hechos se cometan con alguna de las finalidades a las que se refiere el apartado anterior.
\end{enumerate}

\textbf{Artículo 400}

La fabricación, recepción, obtención o tenencia de útiles, materiales, instrumentos, sustancias, datos y programas informáticos, aparatos, elementos de seguridad u otros medios específicamente destinados a la comisión de delitos descritos en los Capítulos anteriores, se castigarán con la pena señalada en cada caso para los autores.


\subsection{TÍTULO XIII. Delitos contra el patrimonio y contra el orden socioeconómico, Capítulo VI de las defraudaciones: Sección 3 de las defraudaciones del fluido eléctrico y análogas}

\textbf{Artículo 256}
El que hiciere uso de cualquier equipo terminal de telecomunicación, sin consentimiento de su titular, ocasionando a éste un perjuicio superior a 400 euros, será castigado con la pena de multa de tres a doce meses.

\textbf{Artículo 560}
\begin{enumerate}[label=\textbf{\arabic*.}]
    \item Los que causaren daños que interrumpan, obstaculicen o destruyan líneas o instalaciones de telecomunicaciones o correspondencia posta, serán castigados con la pena de prisión de uno a cinco años.
\end{enumerate}

\subsection{CAPÍTULO XI. De los delitos relativos a la propiedad intelectual e industrial, al mercado y a los consumidores: SECCIÓN 1. De los delitos relativos a la propiedad intelectual}

\textbf{Artículo 270}
\begin{enumerate}[label=\textbf{\arabic*.}]
    \item Será castigado con la pena de prisión de seis meses a cuatro años y multa de doce a veinticuatro meses el que, con ánimo de obtener un beneficio económico directo o indirecto y en perjuicio de tercero, reproduzca, plagie, distribuya, comunique públicamente o de cualquier otro modo explote económicamente, en todo o en parte, una obra o presentación literaria, artística o científica, o su transformación, interpretación o ejecución artística fijada en cualquier tipo de soporte o comunicada a través de cualquier medio, sin la autorización de los titulares de los correspondientes derechos de propiedad intelectual o de sus cesionarios.
    \item La misma pena se impondrá a quien, en la prestación de servicios de la sociedad de la información, con ánimo de obtener un beneficio económico directo o indirecto, y en perjuicio de tercero, facilite de modo activo y no neutral y sin limitarse a un tratamiento meramente técnico, el acceso o la localización en internet de obras o presentaciones objeto de propiedad intelectual sin la autorización de los titulares de los correspondientes derechos o de sus cesionarios, en particular ofreciendo listados ordenados y clasificados de enlaces a las obras y contenidos referidos anteriormente, aunque dichos enlaces hubieran sido facilitados inicialmente por los destinatarios de sus servicios.
    \item En estos casos, el juez o tribunal ordenará la retirada de las obras o presentaciones objeto de la infracción. Cuando a través de un portal de acceso a internet o servicio de la sociedad de la información, se difundan exclusiva o temporalmente los contenidos objeto de la propiedad intelectual a que se refiere lo.s apartados anteriores, se ordenará la interrupción de la prestación del mismo, y el juez podrá acordar cualquier medida cautelar que tenga por objeto la protección de los derechos de propiedad intelectual.
\end{enumerate}

Excepcionalmente, cuando exista reiteración de las conductas y cuando resulta una medida proporcionada, eficiente y eficaz, se podrá ordenar el bloqueo del acceso correspondiente.


\subsection{CAPÍTULO XI. De los delitos relativos a la propiedad intelectual e industrial, al mercado y a los consumidores: SECCIÓN 3. De los delitos relativos al mercado y a los consumidores}

\textbf{Artículo 278}
\begin{enumerate}[label=\textbf{\arabic*.}]
    \item El que, para descubrir un secreto de empresa se apoderase por cualquier medio de datos, documentos escritos o electrónicos, soportes informáticos u otros objetos que se refieran al mismo, o empleare alguno de los medios o instrumentos señalados en el apartado 1 del artículo 197, será castigado con la pena de prisión de dos a cuatro años y multa de doce a veinticuatro meses.
    \item Se impondrá la pena de prisión de tres a cinco años y multa de doce a veinticuatro meses si se difundieren, revelaren o cedieren a terceros los secretos descubiertos.
    \item Lo dispuesto en el presente artículo se entenderá sin perjuicio de las personas que consideran corresponder por el apoderamiento o destrucción de los soportes informáticos.
\end{enumerate}

\textbf{Artículo 510}
\begin{itemize}
    \item[\textbf{3.}] Las penas previstas en los apartados anteriores se impondrán en su mitad superior cuando lo hechos se hubieran llevado a cabo a través de un medio de comunicación social, por medio  de internet o mediante el uso de tecnologías de la información de modo que, aquel se hiciera accesible a un número elevado de personas
\end{itemize}

\textbf{Artículo 578}
\begin{itemize}
    \item[\textbf{2.}] Las penas previstas en el apartado anterior se impondrán en su mitad superior cuando los hechos se hubieran llevado a cabo mediante la difusión de servicios o contenidos accesibles al público a través de los medios de comunicación, internet, o por medios de servicios de comunicaciones electrónicas mediante el uso de tecnologías de la información.
\end{itemize}


\section{Ética hacker}

El término hacker lo acuña un ``grupo de apasionados programadores'' del MIT a principios de los años 60. A mediados de los 80 los medios de comunicación asocian la palabra hacker a criminal informático. % 34

\chapterA{Profesión}

\section{?` Qué significa ser profesional de la informática?}

Según la \gls{rae}, la profesión es ``el empleo, facultad u oficio que alguien ejerce y por el que recibe una retribución''. Esta palabra viene de profesar, como en profesar una religión. Sus orígenes se remontan a la Edad Media, desde profesar se crea la palabra profesor y aparecen los primeros gremios profesionales.

Según Joseph Migga Kizza, ser un profesional de la informática significa:
\begin{enumerate}[label=\textbf{\arabic*.}]
    \item Tener un conjunto de habilidades altamente desarrolladas y un profundo dominio de la profesión.
          \begin{itemize}
              \item Aunque las competencias profesionales se desarrollan a través de largos años de experiencia, esas habilidades tienen que tener detrás una base muy desarrollada de conocimientos adquiridos durante años de educación formal.
              \item No es lo mismo un trabajador cualificade que un profesional.
          \end{itemize}
    \item Autonomía
          \begin{itemize}
              \item Debido a que les profesionales ofrecen productos o servicios, hay siempre una relación entre proveedores (profesionales) y receptores (clientes). Esta relación tiene que ver con el equilibrio de poder.
              \item Les profesionales pueden tener autonomía para cambiar la forma en la que el servicio se presta sin consultar a sus clientes. Esto no significa cambiar las condiciones.
          \end{itemize}
    \item Seguir un código de conducta:
          \begin{itemize}
              \item Código profesional.
              \item Código personal.
              \item Código institucional.
              \item Código comunitario.
          \end{itemize}
\end{enumerate}

En el CS2013 de \gls{acm} dicen que ``Aunque las cuestiones técnicas son centrales al currículum computacional, no constituyen un programa educativo completo de la materia. Los estudiantes deben ser expuestos al gran contexto social de la computación para desarrollar un entendimiento de los asuntos sociales, éticos legales y profesionales relevantes''.

La denominación de los títulos universitarios oficiales vinculados con el ejercicio de la profesión de Ingeniero Técnico de Informática, deberá facilitar la identificación de la profesión y en ningún caso, podrá conducir a error o confusión sobre sus efectos profesionales.
Desde Mayo de 2015 se equipara Ingeniero Técnico en Informática con Graduado en Informática (cualquiera de sus especificaciones) y con Máster en Ingeniería Informática.

Aunque existe el \gls{ccii} no es necesario colegiarse. No está concretada la responsabilidad civil y penal de la profesión y los proyectos de Informática no necesitan ser visados por un Ingeniere.

\gls{boe} núm. 287, de 4 de agosto de 2009, páginas 66699 a 666710:
``Hasta tanto se establezcan las oportunas reformas de la regulación de las profesiones con carácter general en España y, en concreto, la actualización de las mismas previsto en la normativa vigente [...] acuerda establecer las recomendaciones [...] para las memorias de solicitud de títulos oficiales, propuestas por las Universidades, en los ámbitos de Ingeniería  Informática, título de máster, Ingeniería Técnica Informática.''

Según el \gls{boe}, estás son las características de un profesional en informática:
\begin{itemize}
    \item Capacidad de proyectar, calcular, diseñar productos, procesos e instalaciones [...].
    \item Capacidad para la dirección de obras e instalaciones de sistemas informáticos [...].
    \item Capacidad para dirigir, planificar y supervisar equipos multidisciplinares.
    \item Capacidad para el modelado matemático, calculo y simulación en centros tecnológicos y de ingeniería de empresa, particularmente en tareas de investigación, desarrollo e innovación [...].
    \item Capacidad para la elaboración, planificación estratégica, dirección, coordinación y gestión técnica y económica de proyectos [...].
    \item Capacidad para la dirección general, dirección técnica y dirección de proyectos de fabricación de equipos informáticos, con garantía de la seguridad para las personas y bienes, la calidad final de los productos y su homologación.
    \item Capacidad para la aplicación de los conocimientos adquiridos y de resolver problemas en entornos nuevos o poco conocidos dentro de contextos más amplios y multidisciplinares, siendo capaces de integrar estos conocimientos.
    \item Capacidad para comprender y aplicar la responsabilidad ética, la legislación y la deontología profesional.
    \item Capacidad para aplicar los principios de la economía y de la gestión, de recursos humanos y proyectos, así como la legislación, regulación y normalización de la informática.
\end{itemize}

\section{Responsabilidad, seguridad y control}

Una parte importante de la profesión informática consiste en desarrollar software para terceros. Hay que seguir estándares de verificación y validación y un código ético.

Algunos factores de error a tener en cuenta son:
\begin{itemize}
    \item\textbf{Factor humano (Joseph Migga Kizza):}
    \begin{itemize}
        \item Lapsos de memoria y falta de atención.
        \item Presión por finalizar.
        \item Exceso de confianza.
        \item Maldad.
        \item Complacencia: pasar por alto ciertas pruebas y otras medidas de control de errores en aquellas partes del software que se probaron previamente en un producto similar o relacionado.
    \end{itemize}
    \item\textbf{Naturaleza compleja de los programas (Joseph Migga Kizza):}
    \begin{itemize}
        \item Falta de test (hay billones de combinaciones posibles).
        \item ``Programar es muy fácil, no hace falta saber programar''.
        \item Malentendidos en las especificaciones básicas.
    \end{itemize}
\end{itemize}

Está legislado la seguridad de los procesos informáticos y la protección de datos. Está auditado el proceso de generación del software y el \gls{qos}.

Se deben aplicar principios éticos desde el diseño:
\begin{itemize}
    \item El respeto al ser humano, poniéndolo en el centro del diseño, por encima de otros intereses.
    \item Respeto a la autonomía individual y colectiva.
    \item No discriminación, revisión de sesgos, explicabilidad, transparencia.
    \item Respetar la privacidad.
    \item Respetar el honor.
    \item Respetar los \gls{ddaa}:
          \begin{itemize}
              \item Considerar tipo de licencia.
              \item Considerar diferentes opciones de protección de los \gls{ddaa}.
              \item Respetar los \gls{ddaa}.
          \end{itemize}
    \item Prever y tratar de evitar posibles usos fraudulentos y/o delictivos.
\end{itemize}


\section{Opciones profesionales y búsqueda de trabajo}

Los tipos de empleo / empleadores son:
\begin{itemize}
    \item Empresa privadas:
          \begin{itemize}
              \item Grandes empresas, multinacionales, etc.
              \item Pyme.
              \item Start-Ups.
              \item Consultoría.
          \end{itemize}
    \item Autónome.
    \item Cooperativa.
    \item Administración Pública:
          \begin{itemize}
              \item Técnicos del Estado.
              \item Docencia:
                    \begin{itemize}
                        \item Secundaria.
                        \item Formación Profesional.
                        \item Universitaria.
                    \end{itemize}
              \item Investigación.
          \end{itemize}
    \item Asociación, ONG.
\end{itemize}

Algunas cuestiones importantes en entrevistas de trabajo son: ilegalidad de hacer preguntas personales, ilegalidad de discriminar a una persona por cualquier motivo e ilegalidad de pagar un menor sueldo a una mujer que a un hombre por realizar el mismo trabajo.

\section{Ética profesional}

El propósito principal de los códigos de conducta profesional es promover públicamente la imagen de la profesión mediante la especificación del código y cumplimiento del comportamiento ético esperado de sus miembros.

El Código Ético Profesional del \gls{ccii} promueve la ética en la actividad profesional, el uso ético de la informática, proteger los derechos fundamentales y generar aportaciones de interés general.

En el Capítulo Tercero de dicho código se listan los principios fundamentales bajo el artículo 7:
\begin{enumerate}[label=\textbf{\arabic*.}]
    \item\textbf{Honradez.} El ingeniero informático debe ser moralmente íntegro, veraz, legal y diligente, tanto consigo mismo como en relación con los demás.
    \item\textbf{Independencia.} La independencia actual y moral del Ingeniero en Informática, que permanentemente deberá preservar, es condición esencial para el ejercicio de la profesión y constituye la garantía de los intereses de los destinatarios de sus servicios serán tratados con objetividad.
    \item\textbf{Lealtad.} El Ingeniero en Informática debe ser moralmente integro, veraz, leal y dirigente en el desempeño de su función. El Ingeniero en Informática mantendrá siempre una actitud respetuosa, leal, de colaboración y solidaria con los compañeros de profesión, clientes y demás profesionales y observará la mayor deferencia en sus relaciones profesionales, evitando posiciones de conflicto. En las relaciones o colaboraciones interprofesionales debe respetar los principios, metodologías y decisiones que tienen, como propias y específicas, las demás profesiones, aunque conservando en todo caso la libertad de interpretación y aplicación de los propios fines y objetivos.
    \item\textbf{Dignidad.} El Ingeniero en Informática debe actuar conforme a las normas de honor y de dignidad en la profesión. Debe ejercer la profesión con una conducta irreprochable, guiada por la responsabilidad y la rectitud.
    \item\textbf{Legalidad.} El Ingeniero en Informática debe cumplir y velar por el cumplimiento de todo el ordenamiento jurídico de aplicación de su trabajo, así como por el cumplimiento de las normas corporativas, advirtiendo a las partes involucradas de aquellos aspectos que no cumplan la legalidad vigente y denunciando aquellas actuaciones que supongan un riesgo potencial para la sociedad.
    \item\textbf{Intereses del Cliente.} El ingeniero en Informática debe velar por la satisfacción de los intereses del cliente, incluso cuando estos resulten contrapuestos a los suyos propios. Si se viera en una situación de insuperable contradicción contra sus valores éticos o morales podrá no aceptar el trabajo acogiéndose a la objeción de conciencia.
    \item\textbf{Libertad del Cliente.} El Ingeniero en Informática intentará, en la medida de lo posible, no proponer soluciones que puedan suponer una situación de `cliente prisionero'. Asimismo, el Ingeniero en Informática ha de reconocer el derecho del cliente a elegir con libertad a quien contrata, y por tanto, no poner trabas frente a una posible voluntad de cambio de profesional por parte del cliente.
    \item\textbf{Secreto profesional.} El ingeniero en Informática tiene el derecho y el deber de guardar el secreto profesional de todos los hechos y noticias que conozca por razón de su actuación profesional, con sólo excepciones muy limitadas, que se justifiquen moral o legalmente.
    \item\textbf{Igualdad y Función social.} El Ingeniero en Informática debe tener presente en todo momento el carácter de su cometido como servicio a la sociedad, velando por la igualdad tanto social como de género y ha de promover el conocimiento general de la profesión y su aportación al bien público. El Ingeniero en Informática procurará la mayor eficacia de su trabajo en cuanto a conseguir una óptima rentabilidad social y humana de los recursos disponibles.
    \item\textbf{Adecuación de la Tecnología.} El Ingeniero en Informática debe proponer la solución tecnológica que más se adecúe a las necesidades funcionales y tecnológicas del cliente y a su disponibilidad presupuestaria, evitando la imposición de tecnología.
    \item\textbf{Formación y Perfeccionamiento.} El perfeccionamiento profesional y la continua puesta al día de sus conocimientos técnico-científicos y las mejores práctIcas profesionales es una obligación del Ingeniero en Informática, al permitirle garantizar la prestación de unos servicios de calidad a los usuarios. Del mismo modo, el Ingeniero en Informática debe participar en el desarrollo, uso y regulación de estándares profesionales.
    \item\textbf{Libre y Leal Competencia en el Ejercicio de la Profesión.} El Ingeniero en Informática no puede proceder a la captación desleal de clientes, debiendo respetar en todo momento lo dispuesto en las normas que tutelen la leal competencia y absteniéndose de cualquier práctica de competencia ilícita e informando cuando sea posible a un órgano competente o colegio profesional de cualquier conocimiento real de fraude en concurso o de selección, en especial en los referentes a las administraciones públicas.
    \item\textbf{Remuneración.} El Ingeniero en Informática promoverá y velará en lo posible por la remuneración justa de su trabajo, evitando aceptar aquellos que supongan un menoscabo del prestigio de la profesión o incurran en competencia desleal.
    \item\textbf{Entidades Colegiales.} Las entidades colegiales deben ser consideradas como un ámbito de convivencia entre compañeros, órganos de representación y defensa de legítimos intereses profesionales y una garantía de defensa de la sociedad que promueven una prestación profesional competente, digna, eficaz y responsable.
    \item\textbf{Incompatibilidades.} Además de cuando esté legal o reglamentariamente establecido, se entenderá situación de incompatibilidad, cuando exista colisión de derechos o conflictos de intereses que puedan colocar el ejercicio de la función profesional en una posición equívoca, o que implique un riesgo para su independencia. Cuando el Ingeniero en Informática esté incurso en cualquier causa de incompatibilidad, deberá ponerlo en conocimiento del colegio.
    \item\textbf{Respeto a la Naturaleza y Medio Ambiente.} El respeto y la conservación de la naturaleza y el medio ambiente han de estar entre las preocupaciones de los Ingenieros de Informática en todos los aspectos de su actividad. Los profesionales de la Ingeniería de Informática deberán observar una conducta ecológica en el desempeño de su profesión, debiendo actuar y abogar por y para una defensa de la naturaleza, encaminada a la protección y mejora de la calidad de vida, así como a respeto, disfrute y conservación del medio ambiente adecuado.
    \item\textbf{Trabajo en equipo.} El Ingeniero en Informática cuando participe de un trabajo de equipo, conjuntamente con otras profesiones, deberá actuar con pleno sentido de responsabilidad en el área concreta de su intervención. Asimismo, contribuirá con sus conocimientos y experiencia al intercambio de información técnica al objeto de obtener la máxima eficacia en el trabajo conjunto.
    \item\textbf{Responsabilidad Civil.} El Ingeniero en Informática deberá tener cubierta su responsabilidad profesional, en cuantía adecuada a los riesgos que implique.
    \item\textbf{Investigación y Docencia.} El Ingeniero en Informática como investigador no dará a conocer de modo prematuro o sensacionalista nuevos datos insuficientemente contrastados, no exagerará su significado, ni los falsificará o inventará, ni plagiará publicaciones de otros autores y en general no utilizará con poca seriedad y rigor los datos obtenidos. Es obligación del colegio profesional divulgar a los profesionales los nuevos descubrimientos, avances, novedades técnicas que puedan afectar al adecuado ejercicio profesional. El Ingeniero en Informática, cuando su ejercicio profesional desarrolle actividad docente, tiene el deber de velar por la buena calidad de enseñanza de la profesión, haciendo especial mención de los principios éticos y deontológicos, consustanciales con la misma.
    \item\textbf{Objeción de Conciencia.} La responsabilidad y libertad personal del Ingeniero en Informática le faculta para ejercer se derecho a la objeción de conciencia. El Ingeniero en Informática podrá comunicar al colegio profesional su condición de objetor de conciencia a los efectos que se considere procedentes. El colegio le prestará el asesoramiento y la ayuda necesaria.
\end{enumerate}

Para que los códigos de conducta profesional sean eficaces, debe instituir un sistema de cumplimiento, de informes, de procedimientos de audiencia, y de sanciones y apelaciones:
\begin{itemize}
    \item\textbf{Cumplimiento:} se suele utilizar un panel de expertes que se encargan entre otras cosas de revisar y actualizar los códigos, darlos a conocer, recoger las quejas, abrir procesos disciplinarios, etc.
    \item\textbf{Informes:} de violación del código ético.
    \item\textbf{Procedimientos de audiencia:} convendrían que se hagan cerca del lugar de trabajo y que exista una sanción ejemplar si no se acude a la audiencia.
    \item\textbf{Sanciones:} la revocación de certificación, la solicitud de renuncia al puesto, o la suspensión de la profesión.
    \item\textbf{Apelaciones.}
\end{itemize}


\printglossary[title={Glosario}]

% DESCOMENTAR SI SE USAN IMÁGENES
% \let\cleardoublepage\clearpage
% \listoffigures
% \addcontentsline{toc}{chapter}{Índice de figuras}
% \let\cleardoublepage\clearpage

% DESCOMENTAR SI SE USAN TABLAS
% \listoftables
% \addcontentsline{toc}{chapter}{Índice de cuadros}

\chapter*{Licencia de uso del documento}
\addcontentsline{toc}{chapter}{Licencia de uso del documento}

\begin{flushright}
    \copyright  2022 Alejandro Barrachina Argudo - alejandrobarrachina.02@gmail.com.

    \doclicenseThis

    \url{http://creativecommons.org/licenses/by-sa/4.0/legalcode}.
\end{flushright}

\chapter*{Licencia de uso del código fuente}
\addcontentsline{toc}{chapter}{Licencia de uso del código fuente}

Los archivos de código fuente para generar este documento se encuentran en \url{https://github.com/alk222/ELP} bajo la licencia \href{https://www.gnu.org/licenses/gpl-3.0.html}{GPL-3.0}

\end{document}
