\chapterA{Introducción a la Ética y la Legislación}

\section{Introducción a la Ética}
La ética es una parte de la filosofía, la cual implica cuestionarse sore problemas relativos al bien, al deber, a la virtud o al vicio.
Los objetivos de la ética son:
\begin{enumerate}
    \item Llegar a alguna conclusión acerca de lo que es correcto o no
    \item Que la conclusión pueda ser defendida con argumentos.
    \item Justificar las normas que regulan el comportamiento en diferentes ámbitos de la vida
\end{enumerate}


La reflexión es la consideración minuciosa de un asunto, esto implica dedicarle tiempo y no quedarse con la primera impresión. Para llevar a cabo esta consideración minuciosa debemos:
\begin{itemize}
    \item Informarnos: contrastar fuentes y poder citarlas.
    \item Pros y contrastar.
    \item Partes afectadas.
    \item Circunstancias.
    \item Posibles escenarios futuros.
    \item Contrastar: comentar, dialogar, escuchar otros puntos de vista diferentes al propio.
    \item Discernir, analizar
\end{itemize}

Es importante hacerse muchas preguntas para poder reconocer y tener en cuenta a todas las partes implicadas, dirigir esfuerzos a la hora de buscar información, ampliar nuestra visión del tema y compartir o hacer esas preguntas a ortas personas para ampliar nuestra perspectiva.
Los elementos claves para una reflexión ética son:
\begin{enumerate}
    \item Identificar los valores del conflicto.
    \item Empatía: personas/colectivos afectados: ponerse en el lugar de los otros, percibir a los demás como seres sintientes.
    \item Analizar las circunstancias.
    \item Pros, contras y su peso.
    \item Alcance de las consecuencias en el espacio y el tiempo.
    \item Posibles escenarios futuros partiendo del presente.
\end{enumerate}

Tras una reflexión ética debemos llegar a una conclusión clara, directamente relacionada con los argumentos y ejemplos reales utilizados. Esta conclusión debe ser coherente con los argumentos y ejemplos usados. Si esta conclusión está matizada demostrará un correcto proceso de reflexión.

\section{Introducción a la Legislación}
% El derecho es el arte de lo bueno y lo justo 
El derecho es la técniva de dar a cada persona lo que le corresponde. También es un conjunto de normas organizadas, escritas o no, cuyo cumplimiento es obligatorio y pueden ser impuestas de forma coactiva, que sirven para asegurar la pacífica convivencia.

El Código Civil en su art. 1.1 señala como fuentes del Derecho:
\begin{enumerate}
    \item La Ley: norma vigente.
    \item La costumbre: conductas repetidas desde tiempos inmemoriables. Solo rige en defecto de ley aplicable y siempre que no contradiga la moral o al orden público y tiene que ser probada.
    \item Los \gls{pgd}: principios fundamentales. Se aplican en defecto de la Ley o costumbre. Además, su contenido está siempre presente en el ordenamiento jurídico.
\end{enumerate}

Estas fuentes de derecho tienen una jerarquía, lo que hace que una disposición no carezca de valor si contradice a una de rango superior. Ley $>$ costumbre $>$ \gls{pgd}.
La legislación refleja una serie de valores (Derechos, protección) y una serie de conflictos (delitos, sanción, penas).
