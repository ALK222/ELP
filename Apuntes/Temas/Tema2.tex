\chapterA{Privacidad}

\section{Derecho fundamental}

La privacidad es un derecho fundamental recogido en la Constitución Española de 1978 en el Articulo 18:
\begin{enumerate}
    \item Se garantiza el derecho al honor, a la intimidad personal y familiar y a la propia imagen.
    \item El domicilio es inviolable. Ninguna entrada o registro podrá hacerse en él sin consentimiento del titular o resolución judicial, salvo en caso de flagrante delito.
    \item Se garantiza el secreto de las comunicaciones y, en especial, de las postales, telegráficas y telefónicas, salvo resolución judicial.
    \item La ley limitará el uso de la informática para garantizar el honor y la intimidad personal y familiar de los ciudadanos y el pleno ejercicio de sus derechos.
\end{enumerate}

\section{Redes sociales}

Las redes sociales tienen diseños basados en crear adicción (modelo ``Hooked'' de Mil Eyal), con técnicas de persuasión/ manipulación / coacción y extracción de datos del usuario para su monetización.

Los términos y condiciones de una red social recogen datos importantes como:
\begin{itemize}
    \item Copyright
    \item Información que recogen y cómo la almacenan:
          \begin{itemize}
              \item Qué información.
              \item Donde (transferencia de datos fuera de la \gls{ue}).
              \item Durante cuánto tiempo.
              \item Con quién la comparten.
              \item Para qué la utilizan.
          \end{itemize}
    \item Cambios, notificaciones.
    \item Como cerrar una cuenta y que pasa con ella después de cerrarla,
    \item Cookies de rastreo
    \item Censura de contenido
\end{itemize}

Pero normalmente la gente no las lee, ya que son largas y complicadas con mucho texto jurídico. Es importante leer estos términos de servicio para ser conscientes de lo que hace la aplicación con nuestros datos.

Es importante también saber que una clausula o termino ilegal no tiene validez aunque aparezca en los términos de servicio y que toda empresa que opere en la \gls{ue} debe cumplir con el \gls{rgpd}

El auge de las redes sociales a traído también el ciberacoso, que tiene un mayor impacto y difusión que el acoso ``tradicional''. Podemos decir que hay tres tipos de ciberacoso: exclusión, manipulación y hostigamiento, también podemos diferenciar tres partes: persona que acosa, persona que es acosada y persona observadora.

El ciberacoso puede causar una reacción en cadena, ocasionando un efecto de bola de nieve. No hay violencia física pero causa un perjuicio muy grande en la víctima.

Al usar las redes sociales es importante pensar en qué uso damos de ellas, cuánto las usamos y si las usamos de manera correcta. Es importante el impacto de nuestros actos en las redes sociales.
% \section{Filtraciones}
% \section{Vigilancia}
% \section{Criptografía}
\section{\gls{lopd} y \gls{rgpd}}

El \gls{rgpd} se define en el Reglamento (\gls{ue}) \textbf{2026/679} del Parlamento Europeo y del Consejo de 27 de abril de 2016 y es relativo a la protección de las personas físicas en lo que respecta al tratamiento de datos personales y a la libre circulación de estos datos y por el que se deroga la Directiva 95/46/CE.

La \gls{lopd} es una adaptación de este reglamento en nuestro país, cada país miembro tiene su propia adaptación.


El \gls{rgpd} define su entrada en vigor y aplicación en el \textbf{Artículo 99}:
\begin{itemize}
    \item El presente Reglamento entrará en vigor a los veinte días de su publicación en el Diario Oficial de la Unión Europea.
    \item Será aplicable a partir del 25 de mayo de 2018.
\end{itemize}

El presente reglamento será obligatorio en todos sus elementos y directamente aplicable en cada Estado miembro.

El esquema del \gls{rgpd} es:
\begin{enumerate}
    \item 173 consideraciones (28 páginas): por qué, para qué (preámbulo).
    \item Disposiciones Generales.
    \item Ámbito de aplicación: actividades realizadas en la \gls{ue} independientemente de donde realices el tratamiento.
    \item Definiciones
    \item Principios: relativos al tratamiento, al consentimiento, definir categorías especiales de datos personales etc.
    \item Derechos: transparencia, información y acceso, rectificación, oposición + art. 17 $\rightarrow$ Derecho al olvido.
    \item Limitaciones: casos judiciales etc.
    \item Obligaciones de responsables del tratamiento.
\end{enumerate}

\subsection{Consideraciones}

\textbf{1.-} La protección de las personas físicas en relación con el tratamiento de datos personales es un derecho fundamental. El Artículo 8, apartado 1, de la Carta de los Derechos fundamentales de la \gls{ue} y el Artículo 16, apartado 1, del \gls{tfue} establecen que toda persona tiene derecho a la protección de los datos de carácter personal que le conciernan.

\textbf{6.-} La rápida evolución tecnológica y la globalización han planteado nuevos retos para la protección de datos personales. La magnitud de la recogida y del intercambio de datos personales ha aumentado de manera significativa.
La tecnología permite que tanto las empresas privadas como las autoridades públicas utilicen datos personales en una escala sin precedentes a la hora de realizar sus actividades. Las personas físicas difunden un volumen cada vez mayor de información personal a escala mundial.

\textbf{58.-} El principio de transparencia exige que toda información dirigida al público o al interesado sea concisa, fácilmente accesible y fácil de entender, y que se utilice un lenguaje claro y sencillo, y, además, en su caso, se visualice.
Esta información podría facilitarse en forma electrónica, por ejemplo, cuando esté dirigida al público, mediante un sitio web. Ello es especialmente pertinente en situaciones en las que la proliferación de agentes y la complejidad tecnológica de la práctica hagan que sea mas difícil para el interesado saber y comprender si se están recogiendo, por quién y con qué finalidad, datos personales que le conciernen, como es en el caso de la publicidad en línea.

\textbf{83.-} A fin de mantener la seguridad y evitar que el tratamiento infrinja lo dispuesto en el presente Reglamento, el responsable o el encargado deben evaluar los riesgos inherentes al tratamiento y aplicar medidas para mitigarlos, como el cifrado.

Estas medidas deben garantizar un nivel de seguridad adecuado, incluida la confidencialidad, teniendo en cuanta el estado de la técnica y el coste de su aplicación con respecto a los riesgos y la naturaleza de los datos personales que deban protegerse.

\textbf{101.-} En todo caso, la transferencia a terceros países y organizaciones internacionales solo pueden llevarse a cabo de plena conformidad con el presente Reglamento.

Una transferencia solo podría tener lugar si, a reserva de las demás disposiciones del presente Reglamento, el responsable o encargado cumple las disposiciones del presente Reglamento relativas a la transferencia de datos personales a terceros países u organizaciones internacionales.

\subsection{\gls{aepd}}
La \gls{aepd} es el organismo público encargado de velar por el cumplimiento de la \gls{lopd} en España.

La \gls{aepd} lista los siguientes derechos del ciudadano sobre sus datos:
\begin{itemize}
    \item Derecho a conocer: para qué se usan tus datos, plazo de conservación, derecho a presentar una reclamación ante la \gls{aepd} y la existencia de decisiones automatizadas, elaboración de perfiles y sus consecuencias.
    \item Derecho a solicitar al/la responsable: suspensión del tratamiento de tus datos, conservación de tus datos, portabilidad de tus datos a otros proveedores de servicios.
    \item Derecho a rectificar tus datos: cuando sean inexactos o cuando estén incompletos.
    \item Derecho a suprimir tus datos: por tratamiento ilícito, por desaparición de la finalidad que motivó el tratamiento o recogida, cuando revocas tu consentimiento o cuando te opones a que se traten.
    \item Derecho a oposición al tratamiento de tus datos: Por motivos personales, salvo que quien trata tus datos acredite un interés legítimo, cuando el tratamiento tenga por objeto el marketing directo.
\end{itemize}


La \gls{aepd} también ofrece una ``Guía para responsables de tratamiento de datos'' en la cual se establecen unos principios a seguridad

\textbf{El principio de responsabilidad proactiva.} Este principio exige una actitud consciente, diligente y proactiva por parte de las organizaciones frente a todos los tratamientos de datos personales que se lleven a cabo. Este principio requiere que las organizaciones analicen qué datos tratan, con qué finalidades lo hacen y qué tipo de operaciones de tratamiento llevan a cabo. A partir de este conocimiento deben determinar de forma explícita la forma en la que aplicarán las medidas que el \gls{rgpd} prevé, asegurándose de que esas medidas son las adecuadas para cumplir con el mismo y de que pueden demostrarlo ante los interesados y ante las autoridades de supervisión.
Podemos distinguir seis medidas de responsabilidad activa:
\begin{itemize}
    \item Análisis de riesgos
    \item Registro de actividades de tratamiento
    \item Protección de datos desde el diseño y por defecto
    \item Medidas de seguridad
    \item Notificaciones de ``violaciones de seguridad de los datos''
    \item Evaluaciones de impacto sobre la Protección de datos
\end{itemize}

Todos los responsables deberán realizar una valoración del riesgo de los tratamientos que realicen, a fin de poder establecer qué medidas deben aplicar y cómo deben hacerlo. El tipo de análisis variará en función de: el tipo de tratamiento, la naturaleza de los datos, el número de interesados afectados y la cantidad y variedad de tratamientos que una misma organización lleve a cabo.

Responsables y encargados deberán mantener un registro de operaciones de tratamiento en el que se contenga la información que establece el \gls{rgpd} y que contenga cuestiones como:
\begin{itemize}
    \item Nombre y datos de contacto del responsable o corresponsable y del Delegado de Protección de Datos si existiese.
    \item Finalidades del tratamiento.
    \item Descripción de categorías de interesados y categorías de datos personales tratados.
    \item Transferencias internacionales de datos.
\end{itemize}

Están exentas las organizaciones que empleen a menos de 250 trabajadores, a menos que el tratamiento que realicen pueda entrañar un riesgo para los derechos y libertades de los interesados, no sea ocasional o incluya categorías especiales de datos o datos relativos a condenas e infracciones penales.

\textbf{Protección de Datos desde el Diseño y por Defecto.} Estas medidas se incluyen dentro de las que debe aplicar el responsable con anterioridad al inicio del tratamiento y también cuando se esté desarrollando.

Este tipo de medidas reflejan muy directamente el enfoque de responsabilidad proactiva. Se trata de pensar en términos de protección de datos desde el mismo momento en que se diseña un tratamiento, un producto o servicio que implica el tratamiento de datos personales.

Cuando se produzca una violación de seguridad de los datos, el responsable debe notificar a la autoridad de protección de datos competente, a menos que sea improbable que la violación suponga un riesgo para los derechos y libertades de los afectados.

La notificación de la quiebra a las autoridades debe producirse sin dilación indebida y, a ser posible, dentro de las 72 horas siguientes a que el responsable tenga constancia de ella. Los responsables del tratamiento deberán realizar una \gls{eipd} con carácter previo a la puesta en marcha de aquellos tratamientos que sea probable que conlleven un riesgo alto para los derechos y libertades de los interesados.
