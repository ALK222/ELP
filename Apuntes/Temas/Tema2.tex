\chapterA{Privacidad}

\section{Derecho fundamental}

La privacidad es un derecho fundamental recogido en la Constitución Española de 1978 en el Articulo 18:
\begin{enumerate}
    \item Se garantiza el derecho al honor, a la intimidad personal y familiar y a la propia imagen.
    \item El domicilio es inviolable. Ninguna entrada o registro podrá hacerse en él sin consentimiento del titular o resolución judicial, salvo en caso de flagrante delito.
    \item Se garantiza el secreto de las comunicaciones y, en especial, de las postales, telegráficas y telefónicas, salvo resolución judicial.
    \item La ley limitará el uso de la informática para garantizar el honor y la intimidad personal y familiar de los ciudadanos y el pleno ejercicio de sus derechos.
\end{enumerate}

\section{Redes sociales}

Las redes sociales tienen diseños basados en crear adicción (modelo ``Hooked'' de Mil Eyal), con técnicas de persuasión/ manipulación / coacción y extracción de datos del usuario para su monetización.

Los términos y condiciones de una red social recogen datos importantes como:
\begin{itemize}
    \item Copyright
    \item Información que recogen y cómo la almacenan:
          \begin{itemize}
              \item Qué información.
              \item Donde (transferencia de datos fuera de la \gls{ue}).
              \item Durante cuánto tiempo.
              \item Con quién la comparten.
              \item Para qué la utilizan.
          \end{itemize}
    \item Cambios, notificaciones.
    \item Como cerrar una cuenta y que pasa con ella después de cerrarla,
    \item Cookies de rastreo
    \item Censura de contenido
\end{itemize}

Pero normalmente la gente no las lee, ya que son largas y complicadas con mucho texto jurídico. Es importante leer estos términos de servicio para ser conscientes de lo que hace la aplicación con nuestros datos.

Es importante también saber que una clausula o termino ilegal no tiene validez aunque aparezca en los términos de servicio y que toda empresa que opere en la \gls{ue} debe cumplir con el \gls{rgpd}

El auge de las redes sociales a traído también el ciberacoso, que tiene un mayor impacto y difusión que el acoso ``tradicional''. Podemos decir que hay tres tipos de ciberacoso: exclusión, manipulación y hostigamiento, también podemos diferenciar tres partes: persona que acosa, persona que es acosada y persona observadora.

El ciberacoso puede causar una reacción en cadena, ocasionando un efecto de bola de nieve. No hay violencia física pero causa un perjuicio muy grande en la víctima.

Al usar las redes sociales es importante pensar en qué uso damos de ellas, cuánto las usamos y si las usamos de manera correcta. Es importante el impacto de nuestros actos en las redes sociales.
% \section{Filtraciones}
% \section{Vigilancia}
% \section{Criptografía}
\section{\gls{lopd} y \gls{rgpd}}

El \gls{rgpd} se define en el Reglamento (\gls{ue}) \textbf{2026/679} del Parlamento Europeo y del Consejo de 27 de abril de 2016 y es relativo a la protección de las personas físicas en lo que respecta al tratamiento de datos personales y a la libre circulación de estos datos y por el que se deroga la Directiva 95/46/CE.

La \gls{lopd} es una adaptación de este reglamento en nuestro país, cada país miembro tiene su propia adaptación.


El \gls{rgpd} define su entrada en vigor y aplicación en el \textbf{Artículo 99}:
\begin{itemize}
    \item El presente Reglamento entrará en vigor a los veinte días de su publicación en el Diario Oficial de la Unión Europea.
    \item Será aplicable a partir del 25 de mayo de 2018.
\end{itemize}

El presente reglamento será obligatorio en todos sus elementos y directamente aplicable en cada Estado miembro.

El esquema del \gls{rgpd} es:
\begin{enumerate}
    \item 173 consideraciones (28 páginas): por qué, para qué (preámbulo).
    \item Disposiciones Generales.
    \item Ámbito de aplicación: actividades realizadas en la \gls{ue} independientemente de donde realices el tratamiento.
    \item Definiciones
    \item Principios: relativos al tratamiento, al consentimiento, definir categorías especiales de datos personales etc.
    \item Derechos: transparencia, información y acceso, rectificación, oposición + art. 17 $\rightarrow$ Derecho al olvido.
    \item Limitaciones: casos judiciales etc.
    \item Obligaciones de responsables del tratamiento.
\end{enumerate}
