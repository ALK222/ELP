\chapterA{Derechos Digitales}

Los avances digitales tales como la \gls{ia}, realidad virtual y aumentada o la robótica suponen nuevos retos para la legislación:
\begin{itemize}
    \item Protección de datos personales.
    \item Procesamiento de \textit{big data} justo y respetuoso.
    \item Internet de las cosas.
    \item Robótica.
    \item Sistemas de \gls{ia}.
\end{itemize}

El derecho al respeto por la vida privada es una preocupación, ya que muchas aplicaciones \gls{tic} intentan influir en las actitudes y comportamientos de las personas.
Dichas actividades persuasivas afectan a la autonomía de la persona, a su capacidad de autodeterminación y a si libertad de pensamiento y de conciencia.

Los derechos digitales surgen por las limitaciones de los derechos fundamentales ``tradicionales'' en el ámbito digital, son medidas necesarias para garantizar el respeto de los derechos fundamentales en este ámbito.
Algunos de estos derechos son el derecho a que ``me dejen en paz'', el derecho al olvido y el derecho al acceso a Internet(no es lo mismo que acceso gratuito) entre otros.

\section{Libertad de expresión}

La \textbf{libertad de expresión} se recoge en el Artículo 20 de la constitución y es un derecho fundamental.
La libertad de expresión en Internet se mantiene igual, pero con un carácter más global. Cualquier persona puede hacer públicas sus opiniones, puntos de vista etc. con una mayor posibilidad de impacto y difusión que por otros medios.
Se mantienen los mismos límites establecidos por la constitución:
\begin{itemize}
    \item Respeto al honor.
    \item Respeto a la intimidad.
    \item Respeto a la propia imagen.
    \item Respeto a la juventud y a la infancia.
\end{itemize}

En internet hay filtros en las distintas para plataformas para proteger el copyright, aunque podrían utilizarse para retirar contenidos que ``no interesen''. Son muchos los ejemplos de plataformas que usan estos filtros para imponer su ideología y estándares.

El \textbf{anonimato} garantiza poder expresar opiniones sin temor a las represalias. En internet garantiza la libertad de expresión y también el libre intercambio de información y la privacidad y derecho a no ser espiados. El anonimato se declara como parte del derecho a la libertad de expresión (art. 20 \gls{ce}) y como parte del secreto de las comunicaciones (art 18 \gls{ce})
Así, la extensión del secreto de las comunicaciones a las comunicaciones electrónicas, la garantía de un cierto derecho al anonimato cuando se navegue por Internet, se hagan transacciones económicas o se participe políticamente a través de la Red, aparece como uno de los más importantes derechos, a la vez que más discutido, en la actualidad.

La suma de la IP, mas las cookies, más la minería de datos, puede resultar en la identificación de una persona. Para identificar a una persona(por ejemplo al investigar un delito cometido desde una IP) es necesaria la IP y ``medios que pueden ser razonablemente utilizados'' para asegurar la correspondencia IP-persona.
La protección otorgada a las direcciones IP constituye, por lo tanto, un elemento esencial para mantener el anonimato en Internet. Grandes empresas de Internet (como Google) han cuestionado que la IP sea un dato de carácter personal. Una misma IP puede ser compartida por diferentes usuarios de un mismo \gls{isp} (IPs dinámicas).

\section{Transparencia}

La relación entre publicidad y privacidad o los derechos de acceso a la información, a la intimidad y a la protección de datos es potencialmente conflictiva.
Convergen en un punto de conexión, la divulgación por las autoridades públicas de información que contienen datos personales, lo que quiere dilucidar cuál es la normativa aplicable y las determinaciones sustantivas, procedimentales, de garantías y organizativas que permitan maximizar la eficacia de ambos derechos. Y, a demás, hacerlo de forma adaptar al mundo digital en que actualmente vivimos.

Sobre esta problemática particular se presenta el conflicto entre publicidad y privacidad de la información pública en Internet, y a falta de Autoridades de transparencia y acceso a la información, el protagonismo lo está ejerciendo la \gls{aepd}, que ha dictado resoluciones y recomendaciones del mayor interés sobre esta materia.

\section{Neutralidad de la red}

Todos los paquetes que viajan por la red deben recibir el mismo tratamiento por parte de los \gls{isp} y los gobiernos, no se privilegia a ningún participante por encima de otro.
No se debe cobrar diferente en función del contenido al que se acceda, plataforma, aplicación o tipo de equipamiento utilizado para el acceso.
Esto es muy importante porque garantiza la igualdad de acceso a contenidos de Internet y porque garantiza la privacidad de la información que viaja por la red (que tendría que ser examinada para ser tratada de diferente forma).

\section{Criptografía: derecho fundamental}

Para mantener la privacidad y el anonimato es importante encriptar ciertos tipos de información:
\begin{itemize}
    \item Comunicaciones personales.
    \item Transacciones monetarias.
    \item Contraseñas, números de tarjetas de crédito etc.
    \item Información empresarial.
\end{itemize}

Hay un debate sobre si los gobiernos deberían tener acceso a datos encriptados (Estados Unidos contra Apple, Rusia contra Telegram).

Las leyes sobre criptografía tienen algunas restricciones:
\begin{itemize}
    \item\textbf{Control de exportaciones:} que es la restricción a exportar métodos de criptografía desde un país a otro país o entidad comercial. Hay acuerdos de exportación internacionales, siendo el principal el Acuerdo de Wassenaar.
    \item\textbf{Control de importaciones:} este punto se refiere a las restricciones de usar ciertos métodos de encriptado en un país.
    \item\textbf{Problemas con patentes.}
    \item En algunas ocasiones una persona puede ser obligada a desencriptar archivos o revelar una clave de encriptado.
\end{itemize}

En \gls{eeuu} es necesario pedir permiso antes de publicar un algoritmo o software de cifrado y tienen una regulación de algoritmos criptográficos fuera de su país, el \gls{ear} parte del International Traffic in Arms Regulation.

\section{Comunidades online / virtuales}

Se denomina comunidad virtual a aquella cuyos vínculos, interacciones y relaciones tienen lugar, no en un espacio físico sino en un espacio como Internet.
Las comunidades online se forman a partir de intereses similares entre grupos de personas. Se organizan y se llevan a cabo a partir de objetivos específicos.
Las comunidades saben que son redes, evolucionan de este modo, ampliando los miembros, diversificándose entre sí, nacen en el ciberespacio.

Podemos ver comunidades centralizadas y distribuidas, gobernadas por Empresas, gobiernos o autogestionadas.
