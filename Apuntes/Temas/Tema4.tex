\chapterA{Brecha Digital y Privilegios}
\section{Privilegios y desigualdad}

\epigraph{Los españoles son iguales ante la ley, sin que pueda prevalecer discriminación alguna por razón de nacimiento, raza, sexo, religión, opinión o cualquier otra condición o circunstancia personal o social}{\textit{Artículo 14 \gls{ce}}}

En la sociedad podemos ver una igualdad de derechos, pero una desigualdad de oportunidades.

Debemos ver los privilegios de dos maneras:
\begin{itemize}
    \item Con relación a la ética: con la capacidad de tener en cuenta a todas las partes implicadas en un conflicto y ponerse en su lugar.
    \item En relación con la brecha digital: plantearse si el acceso a la tecnología es o no un privilegio y si tendría que considerarse un Derecho Fundamental.
\end{itemize}

\section{Brecha Digital}

El \gls{desi} es un informe anual publicado por la Comisión Europea que supervisa los avances de los Estados Miembros de la \gls{ue} en el ámbito digital. En este ranking España supera la media europea de personas con capacidades digitales básicas ($64\%$ frente a un $54\%$), pero está por debajo de la media en la proporción de especialistas y titulados en \gls{tic}.

Se consideran competencias digitales básicas saber que existe el correo electrónico y nivel medio al saber utilizarlo.


En España, según datos del \gls{ine}, hay un $1,6\%$ de personas sin habilidades digitales, un $31,4\%$ con habilidad baja y un $19,1\%$ con habilidades básicas, lo que implica que más de la mitad de la población no sepa usar el correo electrónico.

Tenemos que plantearnos si es un verdadero avance estar más conectados sin resolver los problemas éticos relacionados con el diseño como con el uso de la tecnología digital, como si es ético a cada vez obligar a realizar más trámites administrativos por internet.

\section{Brecha de género}

Podemos ver esta brecha de género en las \gls{tic} extrapolando datos de nuestra propia facultad:
\begin{itemize}
    \item\textbf{Curso 2018/2019}:
    \begin{itemize}
        \item $15,3\%$ mujeres(total)
        \item Nuevos ingresos:  $14\%$
    \end{itemize}
    \item\textbf{Curso 2020/2021}:
    \begin{itemize}
        \item $19,4\%$ mujeres(total)
        \item Nuevos ingresos:  n/s
    \end{itemize}
    \item\textbf{Curso 2021/2022}:
    \begin{itemize}
        \item $19,3\%$ mujeres(total)
        \item Nuevos ingresos:  $18\%$
    \end{itemize}
\end{itemize}

Algunas de las posibles causas de esta brecha cuantitativa son:
\begin{itemize}
    \item Falta de referentes femeninos en la profesión.
    \item Percepción de menor capacidad de las mujeres.
    \item Falta de interés natural de las mujeres.
    \item Programas educativos / diseño de entorno muy masculino que no resulta atrayente.
    \item Estereotipos.
    \item Publicidad relacionada con la informática dirigida a público masculino.
\end{itemize}

Las primeras programadoras eran mujeres(``Top secret rosies'':Kathleen
McNulty Mauchly, Marlyn Wescoff Meltzer, Betty Snyder Holberton, Jean
Jennings Bartik, Frances Bilas Spencer y Ruth Lichterman Teitelbaum ), los primeros avances en el software los realizaron mujeres: el primer compilador, primeros lenguajes de alto nivel, primer procesador de texto y el propio termino ``bug''.

A finales de los años 60 y principios de los 70 el software comienza a tener un valor económico y se empiezan a cotizar más los puestos de programación ya que estaban mejor pagados. Esto llevó a que la presencia de las mujeres en la profesión disminuyese al $25\%$.

La falta de presencia de mujeres en las \gls{tic}  tiene como consecuencia que un $51\%$ de la población esté infra-representada en el diseño de la sociedad actual y futura. La falta de diversidad conlleva una peor calidad, menor innovación, menos ingresos y diseños que reproducen estereotipos y discriminaciones, manteniendo así los sesgos.

En 1997 un estudio demuestra que en Suecia una mujer necesita hasta 2,4 veces más méritos que un hombre para recibir una beca pos-doctoral (\url{https://www.nature.com/articles/387341a0}). En 2012 un experimento demuestra que el mismo currículum atribuido a un hombre recibe mayor valoración que cuando es atribuido a una mujer  (\url{http://www.pnas.org/content/early/2012/09/14/1211286109})

Estos son solo dos ejemplos de decenas que demuestran que en las profesiones relacionadas con las ciencias hay una clara preferencia hacia los hombres, dejando en una injusta desventaja a las mujeres.

También son muchos los casos de empresas que han sido llevadas a juicio por pagar menos a empleadas frente a sus compañeros masculinos con el mismo puesto:
\begin{itemize}
    \item \url{https://eu.usatoday.com/story/tech/2017/09/29/oracle-yet-another-tech-firm-hit-suit-allegedly-paying-women-less-than-men/718471001/}
    \item \url{https://eu.usatoday.com/story/tech/2017/09/14/google-hit-gender-pay-gap-lawsuit-seeking-class-action-status/666944001/}
    \item \url{https://www.mercurynews.com/2019/09/19/google-paid-female-engineering-director-less-demoted-her-for-complaining-gender-discrimination-lawsuit/}
\end{itemize}
