\chapterA{Derechos de autor}

\section{Propiedad intelectual, derechos de autor, \gls{lpi}}

\subsection{Productos objeto y derechos del autor}

El Artículo 10 de la \gls{lpi} define que productos son objeto de propiedad intelectual:
\begin{enumerate}[label=\textbf{\arabic*.}]
    \item Son objeto de propiedad intelectual todas las creaciones originales literarias, artísticas o científicas expresadas por cualquier medio o soporte, tangible o intangible, actualmente conocido o que se invente en el futuro, comprendiéndose entre ellas:
          \begin{enumerate}[label=\textbf{\alph*)}]
              \item Los libros, folletos, impresos, epistolarios, escritos, discursos, alocuciones, conferencias, informes forenses, explicaciones de cátedra y cualesquiera otras obras de la misma naturaleza.
              \item Las composiciones musicales, con o sin letra.
              \item Las obras dramáticas y dramático-musicales, las coreografías, las pantomimas y, en general, las obras teatrales.
              \item Las obras cinematográficas y cualesquiera otras obras audiovisuales.
              \item Las esculturas y las obras de pintura, dibujo, grabado, litografía y las historietas gráficas, tebeos o cómics, así como sus ensayos o bocetos y las demás obras plásticas, sean o no aplicadas.
              \item Los proyectos, planos, maquetas y diseños de obras arquitectónicas y de ingeniería.
              \item Los gráficos, mapas, diseños relativos a la topografía, la geografía y, en general, a la ciencia.
              \item Las obras fotográficas y las expresadas por procedimiento análogo a la fotografía.
              \item Los programas de ordenador
          \end{enumerate}
    \item El titulo de una obra, cuando sea original, quedará protegido como parte de ella.
\end{enumerate}


El Capítulo III de la \gls{lpi} define los derechos morales del autor (resumen):

Pertenecen al autor, no pueden cederse:
\begin{itemize}
    \item Decidir si su obra ha de ser divulgada y en qué forma.
    \item Exigir el reconocimiento de su condición de autor de la obra.
    \item Retirar la obra del comercio, por cambio de sus convicciones intelectuales o morales, previa indemnización de daños y prejuicios a los titulares de derechos de explotación.
\end{itemize}

Pueden cederse:
\begin{itemize}
    \item Obtención de beneficios
    \item Reproducción, distribución, comunicación pública y transformación.
\end{itemize}


\subsection{Dominio público}

En los Artículos 26 y 30 se define el tiempo que una obra está amparada bajo la \gls{lpi}:

\textbf{Artículo 26: Duración y cómputo}\\
Los derechos de explotación de la obra durarán toda la vida del autor y setenta años después de su muerte o declaración de fallecimiento.

\textbf{Artículo 30: Cómputo del plazo de protección}\\
Los plazos de protección establecidos en esta Ley se computarán desde el día 1 de enero del año siguiente al de la muerte o declaración de fallecimiento del autor o a la de la divulgación lícita de la obra, según proceda.

Pasados los 70 años las obras pasan al dominio público.

\subsection{Límites y excepciones}

Algunos artículos ponen excepciones y límites a los derechos del autor:
\begin{itemize}
    \item\textbf{Artículo 31} Reproducciones provisionales y copia privada.
    \item\textbf{Artículo 31 bis} Seguridad, procedimientos oficiales y discapacidades.
    \item\textbf{Artículo 32} Cita e ilustración de la enseñanza.
    \item\textbf{Artículo 33} Trabajos sobre temas de actualidad.
    \item\textbf{Artículo 34} Utilización de bases de datos por el usuario legítimo y limitaciones a los derechos de explotación del titular de una base de datos.
    \item\textbf{Artículo 35} Utilización de las obras con ocasión de informaciones de actualidad y de las situadas en vías públicas.
    \item\textbf{Artículo 36} Cable, satélite y grabaciones técnicas.
    \item\textbf{Artículo 37} Reproducción, préstamo y consulta de obras mediante terminales especializados en determinados establecimientos.
    \item\textbf{Artículo 38} Actos oficiales y ceremonias religiosas.
    \item\textbf{Artículo 39} Parodia.
\end{itemize}

\subsection{Sobre la copia privada}

\textbf{Artículo 31. Reproducciones provisionales y copia privada}
\begin{enumerate}[label=\textbf{\arabic*.}]
    \item No requerirán autorización del autor los actos de reproducción provisional a los que se refiere el artículo 18 que, además de carecer por si mismos de una significación económica independiente, sean transitorios o accesorios y formen parte integrante y esencial de un proceso tecnológico y cuya única finalidad consista en facilitar bien una transmisión en red entre terceras partes por un intermediario, bien una autorización lícita, entendiendo por tal la autorizada por el autor o por la ley.
    \item Sin perjuicio de la compensación equitativa prevista en el artículo 25, no necesita autorización del autor la reproducción, en cualquier soporte, sin asistencia de terceros, de obras ya divulgadas cuando ocurran simultáneamente las siguientes circunstancias, constitutivas del límite legal de la copia privada:
          \begin{enumerate}[label=\textbf{\alph*)}]
              \item Que se lleve a cabo por una persona física exclusivamente para su uso privado, no profesional ni empresarial, y sin fines directa ni indirectamente comerciales.
              \item Que la reproducción se realice a partir de una fuente lícita y que no se vulnere las condiciones de acceso a la obra o presentación.
              \item Que la copia obtenida no sea objeto de una utilización colectiva ni lucrativa, ni de distribución mediante precio.
          \end{enumerate}
    \item Quedan excluidas de lo dispuesto en el anterior apartado:
          \begin{enumerate}[label=\textbf{\alph*)}]
              \item Las reproducciones de obras que se hayan puesto a disposición del público conforme al artículo 20.2.i), de tal forma que cualquier persona pueda acceder a ellas desde el lugar y momento que elija autorizándose, con arreglo a lo convenido por contrato y, en su caso, mediante pago de precio, la reproducción de la obra.
              \item Las bases de datos electrónicas.
              \item Los programas de ordenador, en aplicación de la letra a) del artículo 99.
          \end{enumerate}
\end{enumerate}

\textbf{Artículo 25. Compensación equitativa por copia privada}
\begin{enumerate}[label=\textbf{\arabic*.}]
    \item Las reproducciones de obras divulgadas en forma de libros o publicaciones que a estos efectos se asimilen mediante real decreto, así como de fonogramas, videogramas o de otros soportes sonoros, visuales o audiovisuales, realizada mediante aparatos o instrumentos técnicos no tipográficos, exclusivamente para uso privado, no profesional ni empresarial, sin fines directa o indirectamente comerciales, de conformidad con el artículo 31, apartados 2 y 3, originará una compensación equitativa y única para cada una de las tres modalidades de reproducción mencionadas dirigidas a compensar adecuadamente el perjuicio causado a los sujetos acreedores como consecuencia de las reproducciones realizadas al amparo del límite legal de copia privada.
    \item Serán sujetos acreedores de esta compensación equitativa y única los autores de las obras señaladas en el apartado anterior, explotadas públicamente en alguna de las formas mencionadas en dicho apartado, conjuntamente y, en los casos y modalidades de reproducción en que corresponda, con los editores, los productores de fonogramas y videogramas y los artistas intérpretes o ejecutantes cuyas actuaciones hayan sido fijadas en dichos fonogramas y videogramas. Este derecho será irrenunciable para los autores y los artistas intérpretes o ejecutantes.
\end{enumerate}


\section{Protección legal del software}

\subsection{Referencia histórica}

Se proponen dos vías para la protección del software, o bien por patente de invención o bien por derechos de autor. Se optó finalmente por derechos de autor dado que las patentes no ofrecían la suficiente protección para el software. Un software no puede reunir información suficiente para conocer el ``estado de la técnica'' del mismo, imposibilitando el hacer el examen de novedad y actividad inventiva. La Propiedad Intelectual ofrece protección con un mínimo de formalidades y costos y es la más adecuada para la cantidad de software que se genera.

Un programa de ordenador sería patentable como componente de un procedimiento de fabricación o de un aparato protegido por patente o por modelo de utilidad, y exclusivamente para la aplicación del mismo.

\section{?`Qué se puede proteger?}

Se pueden proteger programas como sistemas operativos, controladores y utilidades, compiladores, bibliotecas y entornos de desarrollo y utilidades, guiones y procedimientos almacenados, servidores web y de aplicaciones y código empotrado, firmware y microcódigo.

También se puede proteger la documentación de dichos programas: documentos de análisis de requisitos y de diseño del sistema, planes de pruebas, manuales de instalación y usuario, manuales de referencia, y ayuda interactiva.

No se puede proteger ni ideas, ni algoritmos, ni fórmulas matemáticas, ni principios generales ni interfaces de usuario o de aplicación.

\section{?`A quién pertenece el software?}

\textbf{Artículo 96. Objeto de la protección}

\begin{itemize}
    \item[\textbf{1.}] A los efectos de la presente Ley se entenderá por programa de ordenador toda secuencia de instrucciones o indicaciones destinadas a ser utilizadas, directa o indirectamente, en un sistema informático para realizar una función o una tarea o para obtener un resultado determinado, cualquiera que fuere su forma de expresión y fijación.\\
        A los mismos efectos, la expresión programas de ordenador comprenderá también su documentación preparatoria. La documentación técnica y los manuales de uso de un programa gozaran de la misma protección que este Título dispensa a los programas de ordenador.

    \item[\textbf{4.}] No estarán protegidos mediante los derechos de autor con arreglo a la presente Ley las ideas y principios en los que se basan cualquiera de los elementos de un programa de ordenador, incluidos los que sirven de fundamento a sus interfaces.

\end{itemize}

\textbf{Artículo 97. Titularidad de los derechos}
\begin{enumerate}[label=\textbf{\arabic*.}]
    \item Será considerado autor del programa de ordenador la persona o grupo de personas naturales que lo hayan creado, o la persona jurídica que sea contemplada como titular de los derechos de autor en los casos expresamente previstos por esta Ley.
    \item Cuando se trate de una obra colectiva tendrá la consideración de autor, salvo pacto de lo contrario, la persona natural o jurídica que la edite y divulgue bajo su nombre.
    \item Los derechos de autor sobre un programa de ordenador que sea resultado unitario de la colaboración entre varios autores serán propiedad común y corresponderán a todos éstos en la proporción que determinen.
    \item Cuando un trabajador asalariado cree un programa de ordenador, en el ejercicio de las funciones que le han sido confiadas o siguiendo las instrucciones de su empresario, la titularidad de los derechos de explotación correspondientes al programa de ordenador así creado, tanto el programa fuente como el programa objeto, corresponderán, exclusivamente, al empresario, salvo pacto en contrario.
    \item La protección se concederá a todas las personas naturales y jurídicas que cumplan los requisitos establecidos en esta Ley para la protección de los derechos de autor.
\end{enumerate}


\section{Límites a los derechos de explotación}

\textbf{Artículo 100. Límites a los derechos de explotación}
\begin{itemize}
    \item[\textbf{1.}] No necesitarán autorización del titular, salvo disposición contractual en contrario, la reproducción o transformación de un programa de ordenador incluida la corrección de errores, cuando dichos actos sean necesarios para la utilización del mismo por parte del usuario legítimo, con arreglo a su finalidad propuesta.
    \item[\textbf{2.}] La realización de una copia de seguridad por parte de quien tiene derecho a utilizar el programa no podrá impedirse por contrato en cuanto resulte necesaria para dicha utilización.
    \item[\textbf{3.}] El usuario legítimo de la copia de un programa estará facultado para observar, estudiar o verificar su funcionamiento sin autorización previa del titular, con el fin de determinar las ideas y principios implícitos en cualquier elemento del programa, siempre que lo haga durante cualquiera de las operaciones de carga, visualización, ejecución, transmisión o almacenamiento del programa que tiene derecho a hacer.
    \item[\textbf{5.}] No será necesaria la autorización del titular del derecho cuando la reproducción del código y la traducción de su forma en el sentido de los párrafos a) y b)  del artículo 99 de la presente LEy, sea indispensable para obtener la información necesaria para la interoperabilidad de un programa creado de forma independiente con otros programas, siempre que se cumplan los siguientes requisitos:
        \begin{enumerate}[label=\textbf{\alph*)}]
            \item Que tales actos sean realizados por el usuario legítimo o por cualquier otra persona facultada para utilizar una copia del programa, o, en su nombre, por parte de una persona debidamente autorizada.
            \item Que la información necesaria para conseguir la interoperabilidad haya sido puesta previamente y de manera fácil y rápida, a disposición de las personas a que se refiere el párrafo anterior.
            \item Que dichos actos se limiten a aquellas partes del programa original que resulten necesarias para conseguir la interoperabilidad.
        \end{enumerate}
\end{itemize}

\section{Protección ``sui generis'' de las Bases de Datos}

\textbf{Artículo 12. Colecciones. Bases de datos.}

\begin{enumerate}[label=\textbf{\arabic*.}]
    \item También son objeto de propiedad intelectual, en los términos del libro I de la presente Ley, las colecciones de obras ajenas, de datos o de otros elementos independientes como las antologías y las bases de datos que por la sección o disposición de sus contenidos construyan creaciones intelectuales, sin perjuicio, en su caso, de los derechos que pudieran subsistir sobre dichos contenidos.\\
          La protección reconocida en el presente artículo a estas colecciones se refiere únicamente a su estructura en cuanto forma de expresión de la selección o disposición de sus contenidos, no siendo extensiva a estos.
    \item A efectos de la presente LEy, y sin perjuicio de lo dispuesto en el apartado anterior, se consideran bases de datos las colecciones de obras, de datos o de otros elementos independientemente dispuestos de manera sistemática o metódica y accesibles individualmente por medios electrónicos o de otra forma.
    \item La protección reconocida a las ases de datos en virtud del presente artículo no se aplicará a los programas de ordenador utilizados en la fabricación o en el funcionamiento de bases de datos accesibles por medios electrónicos.
\end{enumerate}

\section{Copyright}

Cuando una obra está protegida bajo copyright no está permitida la reproducción total o parcial de la misma, ni su tratamiento informático, ni la transmisión de ninguna forma o cualquier medio, ya sea electrónico, mecánico, por fotocopia, por registro u otros métodos, sin el permiso previo y por escrito de los titulares del copyright. Reserva también todos los derechos, incluido el derecho a venta, alquiler, préstamo o cualquier otra forma de cesión del uso del ejemplar.


\section{Patentes}

Una patente es un título que reconoce el derecho de explotar en exclusiva la invención patentada, impidiendo a otros su fabricación, venta o utilización sin consentimiento del titular. Como contrapartida, la patente se pone a disposición del público para generar conocimiento.

La patente puede referirse a un procedimiento nuevo, un apartado nuevo, un producto nuevo o un perfeccionamiento o mejora de los mismos. La duración de una patente es de veinte años a contar desde la fecha de presentación de la solicitud. Para mantenerla en vigor es preciso pagar tasas anuales a partir de su concesión.
