\chapterA{Derechos de autor}

\section{Propiedad intelectual, derechos de autor, \gls{lpi}}

\subsection{Productos objeto y derechos del autor}

El Artículo 10 de la \gls{lpi} define que productos son objeto de propiedad intelectual:
\begin{enumerate}[label=\textbf{\arabic*.}]
    \item Son objeto de propiedad intelectual todas las creaciones originales literarias, artísticas o científicas expresadas por cualquier medio o soporte, tangible o intangible, actualmente conocido o que se invente en el futuro, comprendiéndose entre ellas:
          \begin{enumerate}[label=\textbf{\alph*)}]
              \item Los libros, folletos, impresos, epistolarios, escritos, discursos, alocuciones, conferencias, informes forenses, explicaciones de cátedra y cualesquiera otras obras de la misma naturaleza.
              \item Las composiciones musicales, con o sin letra.
              \item Las obras dramáticas y dramático-musicales, las coreografías, las pantomimas y, en general, las obras teatrales.
              \item Las obras cinematográficas y cualesquiera otras obras audiovisuales.
              \item Las esculturas y las obras de pintura, dibujo, grabado, litografía y las historietas gráficas, tebeos o cómics, así como sus ensayos o bocetos y las demás obras plásticas, sean o no aplicadas.
              \item Los proyectos, planos, maquetas y diseños de obras arquitectónicas y de ingeniería.
              \item Los gráficos, mapas, diseños relativos a la topografía, la geografía y, en general, a la ciencia.
              \item Las obras fotográficas y las expresadas por procedimiento análogo a la fotografía.
              \item Los programas de ordenador
          \end{enumerate}
    \item El titulo de una obra, cuando sea original, quedará protegido como parte de ella.
\end{enumerate}


El Capítulo III de la \gls{lpi} define los derechos morales del autor (resumen):

Pertenecen al autor, no pueden cederse:
\begin{itemize}
    \item Decidir si su obra ha de ser divulgada y en qué forma.
    \item Exigir el reconocimiento de su condición de autor de la obra.
    \item Retirar la obra del comercio, por cambio de sus convicciones intelectuales o morales, previa indemnización de daños y prejuicios a los titulares de derechos de explotación.
\end{itemize}

Pueden cederse:
\begin{itemize}
    \item Obtención de beneficios
    \item Reproducción, distribución, comunicación pública y transformación.
\end{itemize}


\subsection{Dominio público}

En los Artículos 26 y 30 se define el tiempo que una obra está amparada bajo la \gls{lpi}:

\textbf{Artículo 26: Duración y cómputo}\\
Los derechos de explotación de la obra durarán toda la vida del autor y setenta años después de su muerte o declaración de fallecimiento.

\textbf{Artículo 30: Cómputo del plazo de protección}\\
Los plazos de protección establecidos en esta Ley se computarán desde el día 1 de enero del año siguiente al de la muerte o declaración de fallecimiento del autor o a la de la divulgación lícita de la obra, según proceda.

Pasados los 70 años las obras pasan al dominio público.

\subsection{Límites y excepciones}

Algunos artículos ponen excepciones y límites a los derechos del autor:
\begin{itemize}
    \item\textbf{Artículo 31} Reproducciones provisionales y copia privada.
    \item\textbf{Artículo 31 bis} Seguridad, procedimientos oficiales y discapacidades.
    \item\textbf{Artículo 32} Cita e ilustración de la enseñanza.
    \item\textbf{Artículo 33} Trabajos sobre temas de actualidad.
    \item\textbf{Artículo 34} Utilización de bases de datos por el usuario legítimo y limitaciones a los derechos de explotación del titular de una base de datos.
    \item\textbf{Artículo 35} Utilización de las obras con ocasión de informaciones de actualidad y de las situadas en vías públicas.
    \item\textbf{Artículo 36} Cable, satélite y grabaciones técnicas.
    \item\textbf{Artículo 37} Reproducción, préstamo y consulta de obras mediante terminales especializados en determinados establecimientos.
    \item\textbf{Artículo 38} Actos oficiales y ceremonias religiosas.
    \item\textbf{Artículo 39} Parodia.
\end{itemize}

\subsection{Sobre la copia privada} %Diapo 10
