\chapterA{Delitos Informáticos}

\section{Definición}

Hay una falta de acuerdo sobre la definición jurídica de delitos informáticos. No están reflejados como al en el Código Penal español, sino que se han añadido aspectos relacionados en delitos ya existentes y se ha añadido algún artículo relativo al daño causado a bienes informáticos.

Algunes juristas consideran necesario diferenciar intrusismo informático de delincuencia informática.
Para Esther Morón el intrusismo son``comportamientos de acceso o interferencia no autorizados, de forma subrepticia, a un sistema informático o red de comunicación electrónica de datos y utilización de los mismos sin autorización o más allá de lo autorizado''.

Los equipos informáticos son nuevos bienes jurídicos, se vela por la integridad de la información y del propio equipo, y a demás los bienes jurídicos que pueden ser accedidos y vulnerados por medios informáticos: patrimonio, intimidad, identidad, material con copyright, etc.

\section{Código Penal Español}

\textbf{Ley Orgánica 10/1995, de 23 de noviembre, del Código Penal}

\textbf{Exposición de Motivos}

Si se ha llegado a definir el organismo jurídico como conjunto de normal que regulan el uso de la fuerza, puede entenderse fácilmente la importancia del Código Penal en cualquier sociedad civilizada.

El código penal define los delitos y faltas que constituyen los presupuestos de la aplicación de la forma suprema que puede revestir el poder coactivo del Estado: la pena criminal. En consecuencia, ocupa un lugar preeminente en el conjunto del ordenamiento, hasta el punto que, no sin razón, se ha considerado como una especie de ``Constitución negativa''.

El Código Penal ha de tutelar los valores y principios básicos de la convivencia social.

\subsection{Tipos de delitos informáticos}

\textbf{Delitos informáticos ``puros''}
\begin{itemize}
    \item Delitos contra la intimidad: descubrimiento y revelación de secretos (art. 197, art. 197 bis, art. 197 ter).
    \item Delito de daños, con especial referencia al sabotaje informático (art. 264, art. 264 bis, art. 264 ter y art. 560).
    \item Tecnología destinada a la comisión de delitos: art. 400.
    \item De la consideración de terrorismo: art. 573.2.
    \item Utilización abusiva dde equipos terminales de comunicaciones: art. 256.
\end{itemize}

\textbf{Delitos ``tradicionales'' que se ven agravados por el uso de las \gls{tic}}:
\begin{itemize}
    \item Delitos relativos a la propiedad intelectual e industrial: art.270 y art. 278.
    \item Uso de las \gls{tic} para realizar estafas: art. 248
    \item Uso de las \gls{tic} en delitos relacionados con abuso a menores: art. 187 y art. 189.
    \item Uso de las \gls{tic} para amenazar: art. 169.
    \item Uso de las \gls{tic} para calumniar e injuriar: art. 205.
\end{itemize}

\subsection{TÍTULO X. Delitos contra la intimidad, el derecho a la propia imagen y a la inviolabilidad del domicilio: CAPÍTULO PRIMERO. Del descubrimiento y revelación de secretos.}

\textbf{Articulo 197}
\begin{enumerate}[label=\textbf{\arabic*.}]
    \item El qie, para descubrir los secretos o vulnerar la intimidad de otro, sin su consentimiento, se apodere de sus papeles, cartas, mensajes de correo electrónico o cualesquiera otros documentos o efectos personales, intercepte sus telecomunicaciones o utilice artificios técnicos de escucha, transmisión, grabación o reproducción del sonido o de la imagen, o cualquier otra señal de comunicación, será castigado con las penas de prisión de uno a cuatro años y multa de doce a veinticuatro meses.
    \item Las mismas penas se impondrán al que, sin estar autorizado, se apodere, utilice o modifique, en prejuicio de terceros, datos reservados de carácter personal o familiar de otro que se hallen registrados en ficheros o soportes informáticos, electrónicos o telemáticos, o en cualquier otro tipo de archivo o registro público o privado. Iguales penas se impondrán a quien, sin estar autorizado, acceda por cualquier medio a los mismos y a quien los altere o utilice en perjuicio del titular de los datos de un tercero.
    \item Se impondrá la pena de prisión de dos a cinco años si se difunden, revelan o ceden a terceros los datos o hechos descubiertos o las imágenes captadas a que se refieren los números anteriores.
\end{enumerate}

\textbf{Artículo 197 bis}
\begin{enumerate}[label=\textbf{\arabic*.}]
    \item El que por cualquier medio o procedimiento, vulnerando las medidas de seguridad establecidas para impedirlo y sin estar debidamente autorizado, acceda o facilite a otro el acceso al conjunto o una parte del sistema de información o se mantenga en él en contra de la voluntad de quien tenga el legítimo derecho a excluirlo, será castigado con pena de prisión de seis meses a dos años.
\end{enumerate}

\textbf{Artículo 197 ter}

Será castigado con una pena de prisión de seis meses a dos años o multa de tres a dieciocho meses el que, sin estar debidamente autorizado, produzca, adquiera para su uso, importe o, de cualquier modo, facilite a terceros, con la intención de facilitar la comisión de alguno de los delitos a que se refieren los apartados 1 y 2 del artículo 197 o el artículo 197 bis:
\begin{enumerate}[label=\textbf{\alph*)}]
    \item Un programa informático, concebido o adaptado principalmente para cometer dichos delitos; o
    \item Una contraseña de ordenador, un código de acceso o datos similares que permitan acceder a la totalidad o a una parte de un sistema de información.
\end{enumerate}

\subsection{TÍTULO XIII. Delitos contra el patrimonio y el orden socioeconómico: CAPÍTULO IX. De los daños}

\textbf{Artículo 264}
\begin{enumerate}[label=\textbf{\arabic*.}]
    \item El que por cualquier medio, sin autorización y de manera grave borrase, dañase, deteriorase, alterase, suprimiese o hiciese inaccesibles datos informáticos, programas informáticos o documentos electrónicos ajenos, cuando el resultado producido fuera grave, será castigado con la pena de prisión de seis meses a tres años.
    \item Se impondrá una pena de prisión de dos a cinco años y multa del tanto al décuplo del perjuicio ocasionado, cuando en las conductas descritas concurra alguna de las siguientes circunstancias:
          \begin{enumerate}[label=\arabic*.a]
              \item Se hubiese cometido en el marco de una organización criminal.
              \item Haya ocasionado daños de especial gravedad o afectado a un número elevado de sistemas informáticos.
              \item El hecho hubiera perjudicado gravemente el funcionamiento de servicios públicos esenciales o la provisión de bienes de primera necesidad.
              \item Los hechos hayan afectado al  sistema informático de una estructura crítica o se hubiera creado una situación de peligro grave para la seguridad del Estado, de la \gls{ue} o de un Estado Miembro de la \gls{ue}. A estos efectos se considerará infraestructura crítica a un elemento, sistema o parte de este que sea esencial para el mantenimiento de funciones vitales de la sociedad, la salud, la seguridad, la protección y el bienestar económico y social de la población cuya perturbación o destrucción tendría un impacto significativo al no poder mantener sus funciones.
          \end{enumerate}
          Si los hechos hubieran resultado de extrema gravedad, podrá imponerse la pena superior en grado.
    \item Las penas previstas en los apartados anteriores se impondrán, en sus respectivos casos, en su mitad superior, cuando los hechos se hubieran cometido mediante la utilización ilícita de datos personales de otra persona para facilitarse el acceso al sistema informático o para ganarse la confianza de terceros
\end{enumerate}

\textbf{Artículo 264 bis}

\begin{enumerate}[label=\textbf{\arabic*.}]
    \item Serán castigados con la pena de prisión de seis meses a tres años el que, sin estar autorizado y de manera grave,  obstaculizara o interrumpiera el funcionamiento de un sistema informático ajeno:
          \begin{enumerate}[label=\textbf{\alph*)}]
              \item Realizando alguna de las conductas a que se refiere el articulo anterior;
              \item Introduciendo o transmitiendo datos; o
              \item Destruyendo, dañando, inutilizando, eliminando o sustituyendo un sistema informático, telemático o de almacenamiento de información electrónica.
          \end{enumerate}
\end{enumerate}

\textbf{Artículo 264 ter}

Será castigado con una pena de seis meses a dos años o multa de tres a dieciocho meses el que, sin estar debidamente autorizado, produzca, adquiera para su uso, importe o, de cualquier modo, facilite a terceros, con la intención de facilitar la comisión de alguno de los delitos a que refiere los dos artículos anteriores:
\begin{enumerate}[label=\textbf{\alph*)}]
    \item Un programa informático, concebido o adaptado principalmente para cometer alguno de los delitos a que se refieren los dos artículos anteriores; o
    \item Una contraseña de ordenador, un código de acceso o datos similares que permitan acceder a la totalidad o a una parte de un sistema de información.
\end{enumerate}

\textbf{Artículo 573}
\begin{enumerate}[label=\textbf{\arabic*.}]
    \item Se considerará delito de terrorismo la comisión de cualquier delito grave contra la vida o la integridad física, la libertad, la integridad moral, la libertad e indemnidad sexuales, el patrimonio, los recursos naturales o el medio ambiente, la salud pública, de riesgo catastrófico, incendio, contra la Corona, de atentado y tenencia, trafico y depósito de armas, municiones o explosivos, previstos en el presente Código, y el apoderamiento de aeronaves, buques u otros medios de transporte colectivo o de mercancías, cuando se llevaran a cabo con cualquiera de las siguientes finalidades:
          \begin{enumerate}[label=\arabic*.a]
              \item Subvenir el orden constitucional, o suprimir o desestabilizar gravemente el funcionamiento de las instituciones políticas o de las estructuras económicas o sociales del Estado, u obligar a los poderes públicos a realizar un acto o a abstenerse de hacerlo.
              \item Alterar gravemente la paz pública.
              \item Desestabilizar gravemente el funcionamiento de una organización internacional.
              \item Provocar un estado de terror en la población o en una parte de ella.
          \end{enumerate}
    \item Se considerarán igualmente delitos de terrorismo los delitos informáticos tipificados en los artículos 197 bis y 197 ter y 264 a 264 quater cuando los hechos se cometan con alguna de las finalidades a las que se refiere el apartado anterior.
\end{enumerate}

\textbf{Artículo 400}

La fabricación, recepción, obtención o tenencia de útiles, materiales, instrumentos, sustancias, datos y programas informáticos, aparatos, elementos de seguridad u otros medios específicamente destinados a la comisión de delitos descritos en los Capítulos anteriores, se castigarán con la pena señalada en cada caso para los autores.


\subsection{TÍTULO XIII. Delitos contra el patrimonio y contra el orden socioeconómico, Capítulo VI de las defraudaciones: Sección 3 de las defraudaciones del fluido eléctrico y análogas}

\textbf{Artículo 256}
El que hiciere uso de cualquier equipo terminal de telecomunicación, sin consentimiento de su titular, ocasionando a éste un perjuicio superior a 400 euros, será castigado con la pena de multa de tres a doce meses.

\textbf{Artículo 560}
\begin{enumerate}[label=\textbf{\arabic*.}]
    \item Los que causaren daños que interrumpan, obstaculicen o destruyan líneas o instalaciones de telecomunicaciones o correspondencia posta, serán castigados con la pena de prisión de uno a cinco años.
\end{enumerate}

\subsection{CAPÍTULO XI. De los delitos relativos a la propiedad intelectual e industrial, al mercado y a los consumidores: SECCIÓN 1. De los delitos relativos a la propiedad intelectual}

\textbf{Artículo 270}
\begin{enumerate}[label=\textbf{\arabic*.}]
    \item Será castigado con la pena de prisión de seis meses a cuatro años y multa de doce a veinticuatro meses el que, con ánimo de obtener un beneficio económico directo o indirecto y en perjuicio de tercero, reproduzca, plagie, distribuya, comunique públicamente o de cualquier otro modo explote económicamente, en todo o en parte, una obra o presentación literaria, artística o científica, o su transformación, interpretación o ejecución artística fijada en cualquier tipo de soporte o comunicada a través de cualquier medio, sin la autorización de los titulares de los correspondientes derechos de propiedad intelectual o de sus cesionarios.
    \item La misma pena se impondrá a quien, en la prestación de servicios de la sociedad de la información, con ánimo de obtener un beneficio económico directo o indirecto, y en perjuicio de tercero, facilite de modo activo y no neutral y sin limitarse a un tratamiento meramente técnico, el acceso o la localización en internet de obras o presentaciones objeto de propiedad intelectual sin la autorización de los titulares de los correspondientes derechos o de sus cesionarios, en particular ofreciendo listados ordenados y clasificados de enlaces a las obras y contenidos referidos anteriormente, aunque dichos enlaces hubieran sido facilitados inicialmente por los destinatarios de sus servicios.
    \item En estos casos, el juez o tribunal ordenará la retirada de las obras o presentaciones objeto de la infracción. Cuando a través de un portal de acceso a internet o servicio de la sociedad de la información, se difundan exclusiva o temporalmente los contenidos objeto de la propiedad intelectual a que se refiere lo.s apartados anteriores, se ordenará la interrupción de la prestación del mismo, y el juez podrá acordar cualquier medida cautelar que tenga por objeto la protección de los derechos de propiedad intelectual.
\end{enumerate}

Excepcionalmente, cuando exista reiteración de las conductas y cuando resulta una medida proporcionada, eficiente y eficaz, se podrá ordenar el bloqueo del acceso correspondiente.


\subsection{CAPÍTULO XI. De los delitos relativos a la propiedad intelectual e industrial, al mercado y a los consumidores: SECCIÓN 3. De los delitos relativos al mercado y a los consumidores}

\textbf{Artículo 278}
\begin{enumerate}[label=\textbf{\arabic*.}]
    \item El que, para descubrir un secreto de empresa se apoderase por cualquier medio de datos, documentos escritos o electrónicos, soportes informáticos u otros objetos que se refieran al mismo, o empleare alguno de los medios o instrumentos señalados en el apartado 1 del artículo 197, será castigado con la pena de prisión de dos a cuatro años y multa de doce a veinticuatro meses.
    \item Se impondrá la pena de prisión de tres a cinco años y multa de doce a veinticuatro meses si se difundieren, revelaren o cedieren a terceros los secretos descubiertos.
    \item Lo dispuesto en el presente artículo se entenderá sin perjuicio de las personas que consideran corresponder por el apoderamiento o destrucción de los soportes informáticos.
\end{enumerate}

\textbf{Artículo 510}
\begin{itemize}
    \item[\textbf{3.}] Las penas previstas en los apartados anteriores se impondrán en su mitad superior cuando lo hechos se hubieran llevado a cabo a través de un medio de comunicación social, por medio  de internet o mediante el uso de tecnologías de la información de modo que, aquel se hiciera accesible a un número elevado de personas
\end{itemize}

\textbf{Artículo 578}
\begin{itemize}
    \item[\textbf{2.}] Las penas previstas en el apartado anterior se impondrán en su mitad superior cuando los hechos se hubieran llevado a cabo mediante la difusión de servicios o contenidos accesibles al público a través de los medios de comunicación, internet, o por medios de servicios de comunicaciones electrónicas mediante el uso de tecnologías de la información.
\end{itemize}


\section{Ética hacker}

El término hacker lo acuña un ``grupo de apasionados programadores'' del MIT a principios de los años 60. A mediados de los 80 los medios de comunicación asocian la palabra hacker a criminal informático. El diccionario del argot hacker es ``The Jargon File''originalmente creado en el MIT y Standford.

La palabra hacker originalmente se refería a alguien que hacía muebles con un hacha, a día de hoy tiene distintos siginificados:

\textbf{hacker} n.
\begin{enumerate}[label=\textbf{\arabic*.}]
    \item Una persona que disfruta explorando los detalles de sistemas programables y como extender sus capacidades, al contrario que la mayoría de usuarios que prefieren aprender el mínimo imprescindible.
    \item Una persona que programa con entusiasmo (o incluso de manera obsesiva) o que disfruta la programación en vez de solo teorizar sobre programación.
    \item Una persona capaz de apreciar el valor del hack.
    \item Una persona que es buena en programar rápido.
    \item Un experto en un programa particular, o que trabaje frecuentemente con él o en él.
    \item Un experto o entusiasta de cualquier tipo. Alguien puede ser hacker de astronomía, por ejemplo.
    \item Alguien que disfruta del desafío intelectual de superar o eludir limitaciones de manera creativa.
    \item [deprecado] Una persona maliciosa que intenta descubrir información sensible hurgando por ahí. El termino correcto para esto es cracker.
\end{enumerate}

\textbf{cracker} n.
\begin{itemize}
    \item Alguien que rompe la seguridad de un sistema. Acuñado por hackers en 1985 en defensa el uso incorrecto de hacker por la prensa. Anteriormente se intentó usar `gusano' para este termino pero no tuvo éxito.
    \item El uso de estos dos términos refleja una repulsa hacia el robo y vandalismo perpetrado por los círculos de crackers. Es esperable que cualquier hacker real haya hecho algo de cracking y conozca algunas de las técnicas básicas, pero también se espera que cualquiera que no sea un principiante haya superado el deseo de usarlas excepto por razones prácticas, inmediatas y benignas.
\end{itemize}

\textbf{Ética hacker} n.
\begin{enumerate}[label=\textbf{\arabic*.}]
    \item La creencia de que compartir información es un poderoso bien positivo, y que es el deber ético de los hackers el compartir su habilidad escribiendo código abierto y facilitando acceso a información y recursos computacionales cuando sea posible.
    \item La creencia de que crackear sistemas por diversión y exploración es éticamente correcto siempre y cuando el cracker no cometa ningún robo, vandalismo o brecha de confidencialidad.
\end{enumerate}

\section{Sistemas distribuidos y descentralizados}

La descentralización es el proceso de dispersar o distribuir funciones, poderes, personas o cosas fuera de una localización o autoridad central.

Mientras que la centralización, sobretodo en esferas de gobierno, está muy estudiada y practicada, no hay una definición común o entendimiento de la descentralización.

Los agentes o partes toman decisiones en el mismo nivel y se les llama \textit{peers}. Las decisiones locales pueden entrar en conflicto unas con otras. Los \textit{peers} interaccionan entre ellos y pueden entrar en la red y abandonarla en cualquier momento. El objetivo de la descentralización es la participación, la diversidad, resolver inferencias derivadas de la falta de recursos en la centralización de servicios y la resolución de conflictos por reducción de desigualdades.

Hay distintos modelos de plataformas descentralizadas:
\begin{itemize}
    \item\textbf{Federadas:} varios nodos centrales que se comunican entre sí y les usuaries pueden elegir con cuál interaccionan.
    \item\textbf{Distribuidas:} redes donde no existe el clásico servidor y que se forman a través de ordenadores corrientes.
    \item\textbf{Blockchain:} nacido a partir de la primera moneda digital Bitcoin.
\end{itemize}
