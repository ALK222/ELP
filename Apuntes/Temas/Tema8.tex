\chapterA{Profesión}

\section{?` Qué significa ser profesional de la informática?}

Según la \gls{rae}, la profesión es ``el empleo, facultad u oficio que alguien ejerce y por el que recibe una retribución''. Esta palabra viene de profesar, como en profesar una religión. Sus orígenes se remontan a la Edad Media, desde profesar se crea la palabra profesor y aparecen los primeros gremios profesionales.

Según Joseph Migga Kizza, ser un profesional de la informática significa:
\begin{enumerate}[label=\textbf{\arabic*.}]
    \item Tener un conjunto de habilidades altamente desarrolladas y un profundo dominio de la profesión.
          \begin{itemize}
              \item Aunque las competencias profesionales se desarrollan a través de largos años de experiencia, esas habilidades tienen que tener detrás una base muy desarrollada de conocimientos adquiridos durante años de educación formal.
              \item No es lo mismo un trabajador cualificade que un profesional.
          \end{itemize}
    \item Autonomía
          \begin{itemize}
              \item Debido a que les profesionales ofrecen productos o servicios, hay siempre una relación entre proveedores (profesionales) y receptores (clientes). Esta relación tiene que ver con el equilibrio de poder.
              \item Les profesionales pueden tener autonomía para cambiar la forma en la que el servicio se presta sin consultar a sus clientes. Esto no significa cambiar las condiciones.
          \end{itemize}
    \item Seguir un código de conducta:
          \begin{itemize}
              \item Código profesional.
              \item Código personal.
              \item Código institucional.
              \item Código comunitario.
          \end{itemize}
\end{enumerate}

En el CS2013 de \gls{acm} dicen que ``Aunque las cuestiones técnicas son centrales al currículum computacional, no constituyen un programa educativo completo de la materia. Los estudiantes deben ser expuestos al gran contexto social de la computación para desarrollar un entendimiento de los asuntos sociales, éticos legales y profesionales relevantes''.

La denominación de los títulos universitarios oficiales vinculados con el ejercicio de la profesión de Ingeniero Técnico de Informática, deberá facilitar la identificación de la profesión y en ningún caso, podrá conducir a error o confusión sobre sus efectos profesionales.
Desde Mayo de 2015 se equipara Ingeniero Técnico en Informática con Graduado en Informática (cualquiera de sus especificaciones) y con Máster en Ingeniería Informática.

Aunque existe el \gls{ccii} no es necesario colegiarse. No está concretada la responsabilidad civil y penal de la profesión y los proyectos de Informática no necesitan ser visados por un Ingeniere.

\gls{boe} núm. 287, de 4 de agosto de 2009, páginas 66699 a 666710:
``Hasta tanto se establezcan las oportunas reformas de la regulación de las profesiones con carácter general en España y, en concreto, la actualización de las mismas previsto en la normativa vigente [...] acuerda establecer las recomendaciones [...] para las memorias de solicitud de títulos oficiales, propuestas por las Universidades, en los ámbitos de Ingeniería  Informática, título de máster, Ingeniería Técnica Informática.''

Según el \gls{boe}, estás son las características de un profesional en informática:
\begin{itemize}
    \item Capacidad de proyectar, calcular, diseñar productos, procesos e instalaciones [...].
    \item Capacidad para la dirección de obras e instalaciones de sistemas informáticos [...].
    \item Capacidad para dirigir, planificar y supervisar equipos multidisciplinares.
    \item Capacidad para el modelado matemático, calculo y simulación en centros tecnológicos y de ingeniería de empresa, particularmente en tareas de investigación, desarrollo e innovación [...].
    \item Capacidad para la elaboración, planificación estratégica, dirección, coordinación y gestión técnica y económica de proyectos [...].
    \item Capacidad para la dirección general, dirección técnica y dirección de proyectos de fabricación de equipos informáticos, con garantía de la seguridad para las personas y bienes, la calidad final de los productos y su homologación.
    \item Capacidad para la aplicación de los conocimientos adquiridos y de resolver problemas en entornos nuevos o poco conocidos dentro de contextos más amplios y multidisciplinares, siendo capaces de integrar estos conocimientos.
    \item Capacidad para comprender y aplicar la responsabilidad ética, la legislación y la deontología profesional.
    \item Capacidad para aplicar los principios de la economía y de la gestión, de recursos humanos y proyectos, así como la legislación, regulación y normalización de la informática.
\end{itemize}

\section{Responsabilidad, seguridad y control}

Una parte importante de la profesión informática consiste en desarrollar software para terceros. Hay que seguir estándares de verificación y validación y un código ético.

Algunos factores de error a tener en cuenta son:
\begin{itemize}
    \item\textbf{Factor humano (Joseph Migga Kizza):}
    \begin{itemize}
        \item Lapsos de memoria y falta de atención.
        \item Presión por finalizar.
        \item Exceso de confianza.
        \item Maldad.
        \item Complacencia: pasar por alto ciertas pruebas y otras medidas de control de errores en aquellas partes del software que se probaron previamente en un producto similar o relacionado.
    \end{itemize}
    \item\textbf{Naturaleza compleja de los programas (Joseph Migga Kizza):}
    \begin{itemize}
        \item Falta de test (hay billones de combinaciones posibles).
        \item ``Programar es muy fácil, no hace falta saber programar''.
        \item Malentendidos en las especificaciones básicas.
    \end{itemize}
\end{itemize}

Está legislado la seguridad de los procesos informáticos y la protección de datos. Está auditado el proceso de generación del software y el \gls{qos}.

Se deben aplicar principios éticos desde el diseño:
\begin{itemize}
    \item El respeto al ser humano, poniéndolo en el centro del diseño, por encima de otros intereses.
    \item Respeto a la autonomía individual y colectiva.
    \item No discriminación, revisión de sesgos, explicabilidad, transparencia.
    \item Respetar la privacidad.
    \item Respetar el honor.
    \item Respetar los \gls{ddaa}:
          \begin{itemize}
              \item Considerar tipo de licencia.
              \item Considerar diferentes opciones de protección de los \gls{ddaa}.
              \item Respetar los \gls{ddaa}.
          \end{itemize}
    \item Prever y tratar de evitar posibles usos fraudulentos y/o delictivos.
\end{itemize}


\section{Opciones profesionales y búsqueda de trabajo}

Los tipos de empleo / empleadores son:
\begin{itemize}
    \item Empresa privadas:
          \begin{itemize}
              \item Grandes empresas, multinacionales, etc.
              \item Pyme.
              \item Start-Ups.
              \item Consultoría.
          \end{itemize}
    \item Autónome.
    \item Cooperativa.
    \item Administración Pública:
          \begin{itemize}
              \item Técnicos del Estado.
              \item Docencia:
                    \begin{itemize}
                        \item Secundaria.
                        \item Formación Profesional.
                        \item Universitaria.
                    \end{itemize}
              \item Investigación.
          \end{itemize}
    \item Asociación, ONG.
\end{itemize}

Algunas cuestiones importantes en entrevistas de trabajo son: ilegalidad de hacer preguntas personales, ilegalidad de discriminar a una persona por cualquier motivo e ilegalidad de pagar un menor sueldo a una mujer que a un hombre por realizar el mismo trabajo.

\section{Ética profesional}

El propósito principal de los códigos de conducta profesional es promover públicamente la imagen de la profesión mediante la especificación del código y cumplimiento del comportamiento ético esperado de sus miembros.

El Código Ético Profesional del \gls{ccii} promueve la ética en la actividad profesional, el uso ético de la informática, proteger los derechos fundamentales y generar aportaciones de interés general.

En el Capítulo Tercero de dicho código se listan los principios fundamentales bajo el artículo 7:
\begin{enumerate}[label=\textbf{\arabic*.}]
    \item\textbf{Honradez.} El ingeniero informático debe ser moralmente íntegro, veraz, legal y diligente, tanto consigo mismo como en relación con los demás.
    \item\textbf{Independencia.} La independencia actual y moral del Ingeniero en Informática, que permanentemente deberá preservar, es condición esencial para el ejercicio de la profesión y constituye la garantía de los intereses de los destinatarios de sus servicios serán tratados con objetividad.
    \item\textbf{Lealtad.} El Ingeniero en Informática debe ser moralmente integro, veraz, leal y dirigente en el desempeño de su función. El Ingeniero en Informática mantendrá siempre una actitud respetuosa, leal, de colaboración y solidaria con los compañeros de profesión, clientes y demás profesionales y observará la mayor deferencia en sus relaciones profesionales, evitando posiciones de conflicto. En las relaciones o colaboraciones interprofesionales debe respetar los principios, metodologías y decisiones que tienen, como propias y específicas, las demás profesiones, aunque conservando en todo caso la libertad de interpretación y aplicación de los propios fines y objetivos.
    \item\textbf{Dignidad.} El Ingeniero en Informática debe actuar conforme a las normas de honor y de dignidad en la profesión. Debe ejercer la profesión con una conducta irreprochable, guiada por la responsabilidad y la rectitud.
    \item\textbf{Legalidad.} El Ingeniero en Informática debe cumplir y velar por el cumplimiento de todo el ordenamiento jurídico de aplicación de su trabajo, así como por el cumplimiento de las normas corporativas, advirtiendo a las partes involucradas de aquellos aspectos que no cumplan la legalidad vigente y denunciando aquellas actuaciones que supongan un riesgo potencial para la sociedad.
    \item\textbf{Intereses del Cliente.} El ingeniero en Informática debe velar por la satisfacción de los intereses del cliente, incluso cuando estos resulten contrapuestos a los suyos propios. Si se viera en una situación de insuperable contradicción contra sus valores éticos o morales podrá no aceptar el trabajo acogiéndose a la objeción de conciencia.
    \item\textbf{Libertad del Cliente.} El Ingeniero en Informática intentará, en la medida de lo posible, no proponer soluciones que puedan suponer una situación de `cliente prisionero'. Asimismo, el Ingeniero en Informática ha de reconocer el derecho del cliente a elegir con libertad a quien contrata, y por tanto, no poner trabas frente a una posible voluntad de cambio de profesional por parte del cliente.
    \item\textbf{Secreto profesional.} El ingeniero en Informática tiene el derecho y el deber de guardar el secreto profesional de todos los hechos y noticias que conozca por razón de su actuación profesional, con sólo excepciones muy limitadas, que se justifiquen moral o legalmente.
    \item\textbf{Igualdad y Función social.} El Ingeniero en Informática debe tener presente en todo momento el carácter de su cometido como servicio a la sociedad, velando por la igualdad tanto social como de género y ha de promover el conocimiento general de la profesión y su aportación al bien público. El Ingeniero en Informática procurará la mayor eficacia de su trabajo en cuanto a conseguir una óptima rentabilidad social y humana de los recursos disponibles.
    \item\textbf{Adecuación de la Tecnología.} El Ingeniero en Informática debe proponer la solución tecnológica que más se adecúe a las necesidades funcionales y tecnológicas del cliente y a su disponibilidad presupuestaria, evitando la imposición de tecnología.
    \item\textbf{Formación y Perfeccionamiento.} El perfeccionamiento profesional y la continua puesta al día de sus conocimientos técnico-científicos y las mejores práctIcas profesionales es una obligación del Ingeniero en Informática, al permitirle garantizar la prestación de unos servicios de calidad a los usuarios. Del mismo modo, el Ingeniero en Informática debe participar en el desarrollo, uso y regulación de estándares profesionales.
    \item\textbf{Libre y Leal Competencia en el Ejercicio de la Profesión.} El Ingeniero en Informática no puede proceder a la captación desleal de clientes, debiendo respetar en todo momento lo dispuesto en las normas que tutelen la leal competencia y absteniéndose de cualquier práctica de competencia ilícita e informando cuando sea posible a un órgano competente o colegio profesional de cualquier conocimiento real de fraude en concurso o de selección, en especial en los referentes a las administraciones públicas.
    \item\textbf{Remuneración.} El Ingeniero en Informática promoverá y velará en lo posible por la remuneración justa de su trabajo, evitando aceptar aquellos que supongan un menoscabo del prestigio de la profesión o incurran en competencia desleal.
    \item\textbf{Entidades Colegiales.} Las entidades colegiales deben ser consideradas como un ámbito de convivencia entre compañeros, órganos de representación y defensa de legítimos intereses profesionales y una garantía de defensa de la sociedad que promueven una prestación profesional competente, digna, eficaz y responsable.
    \item\textbf{Incompatibilidades.} Además de cuando esté legal o reglamentariamente establecido, se entenderá situación de incompatibilidad, cuando exista colisión de derechos o conflictos de intereses que puedan colocar el ejercicio de la función profesional en una posición equívoca, o que implique un riesgo para su independencia. Cuando el Ingeniero en Informática esté incurso en cualquier causa de incompatibilidad, deberá ponerlo en conocimiento del colegio.
    \item\textbf{Respeto a la Naturaleza y Medio Ambiente.} El respeto y la conservación de la naturaleza y el medio ambiente han de estar entre las preocupaciones de los Ingenieros de Informática en todos los aspectos de su actividad. Los profesionales de la Ingeniería de Informática deberán observar una conducta ecológica en el desempeño de su profesión, debiendo actuar y abogar por y para una defensa de la naturaleza, encaminada a la protección y mejora de la calidad de vida, así como a respeto, disfrute y conservación del medio ambiente adecuado.
    \item\textbf{Trabajo en equipo.} El Ingeniero en Informática cuando participe de un trabajo de equipo, conjuntamente con otras profesiones, deberá actuar con pleno sentido de responsabilidad en el área concreta de su intervención. Asimismo, contribuirá con sus conocimientos y experiencia al intercambio de información técnica al objeto de obtener la máxima eficacia en el trabajo conjunto.
    \item\textbf{Responsabilidad Civil.} El Ingeniero en Informática deberá tener cubierta su responsabilidad profesional, en cuantía adecuada a los riesgos que implique.
    \item\textbf{Investigación y Docencia.} El Ingeniero en Informática como investigador no dará a conocer de modo prematuro o sensacionalista nuevos datos insuficientemente contrastados, no exagerará su significado, ni los falsificará o inventará, ni plagiará publicaciones de otros autores y en general no utilizará con poca seriedad y rigor los datos obtenidos. Es obligación del colegio profesional divulgar a los profesionales los nuevos descubrimientos, avances, novedades técnicas que puedan afectar al adecuado ejercicio profesional. El Ingeniero en Informática, cuando su ejercicio profesional desarrolle actividad docente, tiene el deber de velar por la buena calidad de enseñanza de la profesión, haciendo especial mención de los principios éticos y deontológicos, consustanciales con la misma.
    \item\textbf{Objeción de Conciencia.} La responsabilidad y libertad personal del Ingeniero en Informática le faculta para ejercer se derecho a la objeción de conciencia. El Ingeniero en Informática podrá comunicar al colegio profesional su condición de objetor de conciencia a los efectos que se considere procedentes. El colegio le prestará el asesoramiento y la ayuda necesaria.
\end{enumerate}

Para que los códigos de conducta profesional sean eficaces, debe instituir un sistema de cumplimiento, de informes, de procedimientos de audiencia, y de sanciones y apelaciones:
\begin{itemize}
    \item\textbf{Cumplimiento:} se suele utilizar un panel de expertes que se encargan entre otras cosas de revisar y actualizar los códigos, darlos a conocer, recoger las quejas, abrir procesos disciplinarios, etc.
    \item\textbf{Informes:} de violación del código ético.
    \item\textbf{Procedimientos de audiencia:} convendrían que se hagan cerca del lugar de trabajo y que exista una sanción ejemplar si no se acude a la audiencia.
    \item\textbf{Sanciones:} la revocación de certificación, la solicitud de renuncia al puesto, o la suspensión de la profesión.
    \item\textbf{Apelaciones.}
\end{itemize}
